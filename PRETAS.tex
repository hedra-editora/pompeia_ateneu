\textbf{Raul dÁvila Pompeia} (Jacuecanga, Angra dos Reis, 1863--Rio de Janeiro, 1895), 
polemista radical, abolicionista e republicano, foi um dos maiores escritores
brasileiros do século \textsc{xix}, e também um dos mais singulares. Formou"-se
em direito em Recife, depois de ter se transferido do Largo São
Francisco, cujo corpo docente era basicamente composto por
escravocratas e monarquistas ultramontanos. Além de \textit{O
Ateneu}, sua obra maior, legou"-nos os romances
\textit{Uma tragédia no Amazonas} (1880),
\textit{As joias da coroa} (1882), os contos
\textit{Microscópicos} (1881), os poemas em
prosa \textit{Canções sem metro} (1900), as páginas de crônicas 
e reflexões recolhidas em \textit{Alma
morta} (1888) e \textit{Prosas esparsas de
Raul Pompeia} (1920--21). Além de ativista político,
romancista e cronista, foi professor da Escola
Nacional de Belas"-Artes, diretor da Biblioteca Nacional, desenhista e
pintor. Os desenhos deste volume são de autoria do próprio autor. Raul
Pompeia suicidou"-se no Rio de Janeiro, no quarto de sua casa, na
noite de Natal de 1895.


\textbf{O Ateneu} foi publicado em capítulos,
no jornal carioca \textit{A gazeta de
notícias}, entre 8 de abril e 18 de maio de 1888, e,
devido ao reconhecimento imediato, foi editado em livro no mesmo ano.
Escrito em apenas três meses, é considerado o maior romance brasileiro do
século \textsc{xix} depois dos romances realistas de Machado de Assis. Seu
enredo consiste na recordação do período de dois anos em que o narrador,
Sérgio, passa num tradicional colégio interno do Rio de Janeiro. O
ingresso no Ateneu marca as descobertas amargas que acompanharão o
narrador daí em diante, os sentimentos de desilusão, opressão e
desconfiança, componentes da profunda solidão humana. Seu sentido é o
de um ritual de passagem, em que o convívio com os colegas, os
professores e o diretor definem a afirmação moral, sexual e intelectual
de um menino de 11 anos. Difíceis de definir, o estilo e o
significado do romance geraram uma das mais profícuas polêmicas da
história da nossa literatura, aqui apresentada e antologizada
cronologicamente no final do volume.


\textbf{Caio Gagliardi} é professor da Universidade de São Paulo na área de Literatura Portuguesa, onde coordena o grupo de pesquisas Estudos Pessoanos. É autor de \emph{O renascimento do autor: autoria, heteronímia e fake memoirs} (Hedra, 2019) e organizador de \emph{Fernando Pessoa \& Cia. não heterônima} (Mundaréu, 2019), entre outras publicações. 



