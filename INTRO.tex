\chapter[Introdução, \emph{por Caio Gagliardi}]{Introdução}
\hedramarkboth{introdução}{caio gagliardi}

\begin{flushright}
\textsc{caio gagliardi}
\end{flushright}

\section*{um homem de honra}

\epigraph{Que importa a inspiração doentia se o resultado é belo? A pérola é a enfermidade de um molusco.}{Raul Pompéia}

Quando publica \textit{O
Ateneu}, aos 25 anos, Raul Pompéia é já um escritor
experiente.\footnote{ É verdade que, se hoje esse dado chama a
atenção, no século \textsc{xix} não causava estranheza. Toda a obra de Álvares
de Azevedo, Junqueira Freire, Casimiro de Abreu e Castro Alves foi
escrita até os 25 anos (a do primeiro, de qualidade notável, até os
20). Mas se, entre poetas, esse era um fenômeno comum, entre prosadores
a morte prematura foi episódica, e a escrita precoce, menos frequente.
A trajetória de Pompéia vai muito além da atividade como romancista,
tendo sido toda ela singularizada pela precocidade.}
Sua estreia nas letras ocorre aos 14 anos, num pequeno jornal do
colégio Abílio,\footnote{O colégio Abílio era a
última palavra em matéria de pedagogia no 2º Reinado.} no Rio de Janeiro, 
em que traduz um estudo em inglês sobre
Michelangelo. Aos 15, transfere"-se para o então famoso Imperial
Colégio D.~Pedro \textsc{ii}, para completar o equivalente hoje ao ensino médio.
Com essa idade já tem um romance escrito, \textit{Uma
tragédia no Amazonas}, que publica em 1880, com recursos do pai.


Apesar da juventude do autor, o livro foi bem recebido
nos jornais cariocas por críticos eminentes, como Capistrano de
Abreu\footnote{ Preferi citar a crítica de Laet porque, embora mais
lembrada, a crítica de Capistrano não é de todo isenta, já que, como se
pode comprovar, o crítico e o escritor eram grandes amigos. Eis o
depoimento de Capistrano: ``Sempre que nos encontrávamos na rua do
Ouvidor, era certo passarmos juntos o resto do dia. Quantas vezes não
sucedia separarmo"-nos à meia"-noite, depois de sete ou oito horas de
ininterrupta conversação sobre assuntos políticos e literários!'' ``Raul
Pompéia como esteta''. In \textit{Teoria, crítica e história literária};
seleção e apresentação de Alfredo Bosi, Edusp, São Paulo, 1978, p.~212.}
e Carlos de Laet: ``Um aperto de mão é o que envio ao
sr.~Pompéia, inteligente mancebo que experimentou suas forças
escrevendo --- \textit{Uma tragédia no
Amazonas} --- romancete esboçado em horas de lazeres
acadêmicos, porém, no qual já se revelam apreciáveis
talentos''.\footnote{ Citado por seu biógrafo, Eloy Pontes, em
\textit{A vida inquieta de Raul Pompéia}, Livraria José
Olympio Editora, Rio de Janeiro, 1935, p.~45.}

A data de publicação do primeiro romance é também a da
grande festa que a Corte, instalada no Rio de Janeiro, preparou para
homenagear o maestro Carlos Gomes, recém"-chegado de uma turnê
bem"-sucedida pela Europa. Entre os festejos: uma orquestra com 400
músicos tocando \textit{O Guarany}, e muitos
discursos. Um dos oradores: o adolescente Raul Pompéia.

O escritor muda"-se para São Paulo, em 1881, para
matricular"-se na Faculdade de Direito do Largo São Francisco, de onde
eram egressos nomes importantes como Álvares de Azevedo, Castro Alves e
Rui Barbosa (os dois últimos, abolicionistas e ex"-alunos do colégio
Abílio). Na nova capital, Pompéia logo funda o \textit{Jornal
 do Comércio} e torna"-se um escritor militante. Ao
lado do jornalista Alberto Torres, do poeta e jornalista Xavier
Silveira, e dos poetas Raymundo Correia e Luiz Murat, encampa a luta
libertária pela abolição da escravatura e proclamação da República. É o
período em que fundam a \textit{Abolicionista Luiz
Gama} (em homenagem ao poeta mulato e orador
libertário, morto em 1882), e passam a proferir conferências em
teatros. São Paulo acende em Pompéia a flama do homem político.
%Bruno: Fundam a... o quê [a gazeta?] ??? Consultar Caio.

Sem ter completado os vinte anos, o escritor é já um nome conhecido na
cidade. Considere"-se que, a duas décadas do século \textsc{xx}, São Paulo era
um lugar pequeno e relativamente tranquilo. Uma discussão aqui, uma
rixa ali; não se precisava muito para que um acontecimento chegasse às
páginas do jornal. O jornalismo da época, por sua vez, era muito
diferente daquele que conhecemos hoje. A figura do jornalista
confundia"-se com a do cronista social. A apuração dos fatos (a
essência do jornalismo moderno) não era considerada mais importante do
que o estilo que os narra. O cronista é sempre opinativo, pessoal e
afeito às pinceladas de ficção. Naquele tempo, o que se discutia nos
cafés, teatros e praças era matéria para essas crônicas. A notícia,
afinal, são muitas vezes as ideias --- os acontecimentos estão ainda ao
alcance dos olhos e dos ouvidos das pessoas. Muitos escritores, que não
encontravam na literatura uma segura fonte de renda, foram absorvidos
pelo jornalismo. Não por acaso, autores de renome, como os irmãos
Aluísio e Artur Azevedo, Olavo Bilac, Lima Barreto, Machado de Assis,
Coelho Neto e o próprio Raul Pompéia foram cronistas sociais.


Nessa cidade de dimensões ainda mensuráveis, ganhou
especial notabilidade um episódio da trajetória acadêmica do escritor.
No exame para o 3º ano do curso
de Direito, ele e Luiz Murat são reprovados. Redator e caricaturista do
jornal \textit{O Bohemio}, Pompéia publicara
uma charge, assinada sob o pseudônimo Rapp, em que ridicularizava o
conservador \textit{Diário de Campinas}. A
charge é uma paródia da \textit{via crucis},
em que a figura de Cristo é substituída pela de um asno, que
simbolizava a estupidez do jornal campineiro. A atitude e o entusiasmo
do aluno provocaram desconforto entre os docentes, em geral
escravocratas, retrógrados e católicos provincianos. Um deles, o
professor Leite Moraes, era inclusive muito ligado ao
\textit{Diário}.


A posição ideológica liberal e a veemência com que se
expressava rendiam desavenças a Raul Pompéia. Aos olhos dos
monarquistas, tornara"-se um ``subversivo'', da aristocracia
escravocrata, ``inconveniente''; e um ``desagradável'' para a mentalidade
acadêmica e burguesa. Contra as autoridades e seus afetos, escreve: ``O
meio termo é o `status quo' da covardia. Na lógica é o pavor da
consequência, desfiada em deduções pelo declive do argumento. Na vida
comum é a duplicidade tímida, ante as coerências enérgicas do
caráter''.\footnote{Apud Eloy Pontes, p.~49.}


Pompéia não apenas tomava partido nas discussões de seu tempo, como
comprava as brigas e compartilhava as dores daqueles com quem
concordava, a ponto de se deixar levar aos extremos das questões. A
leitura de seus textos republicanos e abolicionistas revela o impulso
transgressor de um intelectual que anseia por reformas. Na argumentação
e no tom desses textos, destaca"-se a sátira e mesmo a brutalidade.
Essa contundência verbal, privilegiada nas polêmicas em que se envolve,
fornece os contornos do perfil de um intelectual combativo e, muitas
vezes, intransigente.


Jornais em São Paulo e, inclusive, no Rio de Janeiro
recebem a notícia da reprovação com ironia e sarcasmo. Entre eles,
estavam os próprios \textit{Diário de Campinas} e \textit{Jornal do
Comércio}, diretamente envolvidos na polêmica, além de
\textit{A Gazeta da Tarde}, do Rio, e a
\textit{Gazeta de Notícias}, editado por
Valentim Magalhães. Até o talvez mais famoso cronista da época, Joaquim
Serra, do grupo de Machado de Assis, sai em defesa de Pompéia. Nesse
ambiente carregado de indignação, Magalhães pinta do seguinte modo o
retrato do escritor: 

\begin{hedraquote}
Hoje Raul Pompéia é dos moços mais conhecidos de S.~Paulo, 
popularmente estimado. Redigia o \textit{Jornal do
Comércio} até pouco tempo. Daí, talvez, a causa de sua
\textit{bomba}. Raul é de uma índole
inquieta, entusiástica, cheio de vida, fanático pelo movimento, pela
luz, pelos grandes ruídos. Mas um coração ingênuo, profundamente
sensível.\footnote{Ibid., p.~137.}
\end{hedraquote} 


A situação ficara insustentável. De um lado, os
professores, ridicularizados pela imprensa, do outro os jovens, firmes
no ataque. O resultado foi a reprovação em massa, e um escândalo geral.
Em decorrência, os estudantes entram em greve, abandonam a Academia, e
nada menos que 94 deles se transferem para Recife,\footnote{Repare o
leitor que, naquele tempo, a viagem a Recife era feita de navio. A saga
desses 94 estudantes na capital pernambucana talvez merecesse ser contada. Brito
Broca lembra, por exemplo, do surto de febre amarela que acometeu
alguns deles. In ``Raul Pompéia'', \textit{Ensaios de mão canhestra}, Pólis, 
São Paulo, 1981, pp.~214--220.} onde concluem o curso.


A propensão do autor pelo pensamento radical e
incondicional despertou a suposição da existência de um drama mais
íntimo em sua personalidade, que pudesse ser tomado como justificativa
para episódios de sua biografia e para algumas das páginas mais
contundentes que deixou. Sob a forma de expressões e adjetivos na linha
de ``sensibilidade enferma'', ``nevrótico'', ``irascível'', ``indisciplinado'',
``neurastênico'', ``frágil'', ``receoso'', ``desconfiado'', ``intratável'',
``irritadiço'', ``suscetível'', ``suspeitoso'', ``assomado'' e ``hipersensível'',
uma curiosidade imoral, ainda hoje alimentada como pretexto de uma
escrita autobiográfica em \textit{O Ateneu},
tende a obscurecer a rara figura do intelectual espirituoso e disposto
a empenhar"-se pelo país e pelo povo. É esta, para além das manias,
que nos interessa.

Em 1888, o desenvolvimento progressivo da campanha abolicionista (com a
Lei do Ventre Livre, 1871, e a dos Sexagenários, 1885) culmina com a
proclamação da Lei Áurea, que abolia oficialmente a escravatura.
Paralelamente, a campanha republicana, que tivera início em 1871, com a
realização do primeiro congresso do \textsc{prp} (Partido Republicano Paulista),
em Itu, encerra"-se com a proclamação da República, em 1889. Mas o
período é de forte crise: a política do Encilhamento, com Rui Barbosa
no Ministério da Fazenda, é marcada pela especulação monetária e
consequente inflação.

Pompéia foi um homem de ideias persistentes. Florianista
radical,\footnote{Com a proclamação da República (1889), o marechal
Deodoro da Fonseca assume o governo provisório. Promulgada a primeira
Constituição, a Assembleia Constituinte nomeia"-o presidente, e
escolhe o marechal Floriano Peixoto seu vice. Presidente e vice eram de
chapas diferentes, o que estimula o rompimento interno entre os
republicanos e as querelas civis. Após fechar o Congresso Nacional,
Deodoro, na iminência de um levante civil contra o governo, recebe o
contragolpe de Floriano, que assume a presidência (1891) e restaura a
ordem constitucional.} e desde 1891 professor de
mitologia na Escola Nacional de Belas Artes, passa a publicar críticas
aos ``inimigos da República'', referindo"-se ao grupo do qual faziam
parte velhos amigos, como Luiz Murat, Guimarães Passos, Oscar Rosas e
Olavo Bilac. A este, que publicara um ano antes uma coluna elogiosa a
respeito de Pompéia, é atribuída uma crítica grosseira ao autor,
veiculada no \textit{Vida Fluminense}, em
1892, sob o pseudônimo Pierrot.\footnote{O artigo,
segundo Eloy Pontes, poderia ter sido escrito por Rosas, na ausência de
Bilac, afetado pela habitual ressaca --- eco matutino da vida boêmia.
Bilac, no entanto, nunca esclareceu a questão da autoria, e o silêncio
significou consentimento. Acusando Pompéia de servilismo, adulação e
pretensão, o artigo termina com a seguinte grosseria: ``Talvez não seja
pretensão, talvez seja amolecimento cerebral, pois que Raul Pompéia
masturba"-se e gosta de, altas horas da noite, numa cama fresca, à
meia"-luz de \textit{veilleuse} mortiça,
recordar, amoroso e sensual, todas as beldades que viu durante o seu
dia, contando em seguida as tábuas do teto onde elas vaporosamente
valsam''. Ibid., p.~242.} A réplica vem na mesma
moeda.\footnote{ Em sua réplica, Pompéia escreve:
``Quanto a responder\ldots{} haveria mister voltar contra os agressores a
mesma arma fácil de afronta, de que se serviram, assacar um doesto bem
forte, dizer, por exemplo, detidamente, que o ataque foi bem digno de 
uns tipos, alheados do respeito humano, licenciados, marcados, sagrados 
--- para tudo --- pelo estigma preliminar do Incesto''. Ibid., p.~243.} 
Ambos os escritores chegam a trocar desaforos, e
Pompéia desafia Bilac para um duelo de espadas, que, graças à
interferência de terceiros, não chega a acontecer. 

O clima de exaltação inflamava o sentimento nacionalista. Floriano, o
``Marechal de Ferro'', apaziguava revoltas, como a da Armada, no Rio, e a
revolução federalista, no Sul. Em 1894, Pompéia é nomeado diretor da
Biblioteca Nacional, e Prudente de Morais assume a presidência como o
primeiro presidente civil brasileiro. A República se consolida, mas não
aplaca as divergências internas entre seus adeptos. No ano seguinte
morre Floriano. Ele é enterrado provisoriamente, enquanto a população
aguarda durante três meses a transferência do caixão para um mausoléu
de mármore, especialmente construído no cemitério São João Batista.

O sepultamento definitivo, em 29 de setembro, é uma
cerimônia bastante solene, com a presença de Prudente de Moraes,
ministros e comandantes militares. Muitos aproveitaram o momento para
falar. O enterro transformou"-se num comício, com ironias e ataques ao
regime. Em clima de exaltação, Raul Pompéia tomou a palavra,
provavelmente quando o presidente e a comitiva já haviam se ausentado.
Foi preciso a interferência da polícia montada, dentro do cemitério,
para debandar o grupo. No primeiro despacho coletivo do Governo,
Pompéia era exonerado do cargo de diretor da Biblioteca Nacional. No
dia 5 de outubro, explicava"-se no \textit{Paiz}:

\begin{hedraquote}
É absolutamente falso que eu houvesse proferido a mínima
palavra de ofensa pessoal a qualquer autoridade da República. O meu
discurso fez inteira exclusão de personalidade e versou sobre
proposições teóricas da política, leal e francamente exibidas, conforme
é meu costume, e ouvidas atentas pelos principais personagens
presentes, na imensa assembleia --- até a última fase.\footnote{Ibid., p.~272.} 
\end{hedraquote}


Cerca de um mês depois da demissão, é publicada uma crônica provocativa
do ex"-colega Luiz Murat, intitulada ``Um louco no cemitério'',
defendendo a punição exemplar de Pompéia. O texto acusa"-o ainda de
adulador do regime militar, ideologicamente incoerente, demagogo,
vingativo e covarde. O patriotismo de Pompéia, matizado pelo
florianismo intransigente, adquirira feições místicas. Proclamada a
República, não depusera as armas. Pelo contrário, tornara"-se um homem
inteiramente absorvido pela política e intransigente com relação ao
governo.

As acusações de Murat ficaram sem resposta. Pompéia
interpretara como conspiração o atraso na publicação de uma crônica
sua, em \textit{A Notícia}, sentindo"-se
completamente desprestigiado. Em casa, se diz desonrado e ameaça
suicidar"-se. Seu último documento é um bilhete, escrito imediatamente
antes de morrer: ``À \textit{Notícia} e ao
Brasil, declaro que sou homem de honra''.\footnote{Em suas crônicas, 
Pompéia refletiu repetidas vezes a respeito do suicídio.
Tratava o assunto sem gravidade, em tom quase sempre irônico. No
\textit{Jornal do Comércio}, em 1889,
escreveu: ``O homem é o animal capaz de suicídio. Tem"-se logo esta
definição que é a melhor apologia dos suicidas (que lhes renda bom
proveito). Acumulada uma quantidade de pesares que cada um avalia como
entende, o homem tem o direito de cortar; basta!'' Ibid., p.~312.}

Raul Pompéia cometeu suicídio na noite de Natal de 1895, com um tiro no peito.
\asterisc

Referindo"-se ao episódio trágico, Lêdo Ivo
responsabiliza ``um tal de Luís Murat que, não tendo legado à
posteridade nem sequer um ponto"-e"-vírgula, foi o odioso responsável
pelo seu suicídio, causado por um imperativo de honra pessoal
ferida''.\footnote{Lêdo Ivo, \textit{O universo poético de Raul
Pompéia}, Livraria São José, Rio de Janeiro, 1963, p.~68.} 
De fato, Murat não foi mais que um poeta
romântico tardio, um tanto supervalorizado em seu tempo. Para Agrippino
Grieco, Pompéia foi ``uma vítima da desproporção entre os seus
entusiasmos e a mediocridade do ambiente''. Segundo o crítico, ``um tal
homem não poderia permanecer longo tempo no mundo''.\footnote{Agrippino Grieco, 
\textit{Evolução da prosa brasileira}, Ariel, Rio de Janeiro, 1933, p.~102.}  

Nos anos que separam o romance de estreia do autor de
sua obra maior, ganham as páginas dos jornais, além de muitas crônicas,
a novela \textit{As jóias da coroa} (na
\textit{Gazeta de Notícias}), o conjunto de
contos \textit{Microscópicos} (na
\textit{Comédia}), e o de poemas em prosa
\textit{Canções sem metro} (no
\textit{Jornal do Comércio}). Escreve também
\textit{Agonia}, um romance inacabado,
deixado em manuscrito, e deixa ainda páginas de reflexões e
confidências, de caráter corrosivo e escritas em forma de cartas,
\textit{Alma morta} (na \textit{Gazeta da Tarde}) --- depois retomada
como \textit{Cartas para o futuro}. O
caráter conjectural desses textos e o trabalho constante com o estilo,
através de numerosas revisões, são ingredientes do fermento estilístico
deste livro.  

\textit{O Ateneu} foi escrito
em apenas três meses. Como era comum na época, sua primeira publicação
ocorreu em folhetins, isto é, em capítulos no jornal carioca
\textit{A gazeta de notícias}, entre 8 de
abril e 18 de maio de 1888. O reconhecimento imediato e a recepção
positiva do romance impulsionaram, no mesmo ano, sua edição em livro.

Muitos críticos o tratam como um fenômeno literário:
``Raul Pompéia, que se soterrou ele próprio em plena mocidade, em dias
de plena criação artística, escreveu aos vinte e quatro anos um livro
simplesmente miraculoso para um adolescente dos trópicos''.\footnote{Ibid, p.~101.} 
Outros vão mais além, e consideram
\textit{O Ateneu} o romance brasileiro do
século \textsc{xix} mais próximo em qualidade dos romances realistas de Machado
de Assis. Segundo José Paulo Paes, o livro ``é um dos momentos mais
altos da ficção brasileira, só comparável [\ldots{}] à epifania do romance
machadiano''.\footnote{José Paulo Paes, ``Sobre as ilustrações
d'\textit{O Ateneu}'', em \textit{Gregos e Baianos}, Brasiliense, 
São Paulo, 1985, p.~51.} Para José Guilherme
Merquior, Pompéia é ``nosso maior romancista impressionista depois de
Machado de Assis''.\footnote{José Guilherme Merquior, em \textit{De
Anchieta a Euclides --- breve história da literatura brasileira}, José Olympio, 
Rio de Janeiro, 1977, p.~191.} A propósito
dessa comparação, pesa o fato de \textit{Memórias póstumas de
Brás Cubas} (1881), título que abre a chamada fase
madura de Machado, ter sido escrito quando o autor já passara dos
quarenta anos.


Como se não bastasse, Pompéia não se ateve ao exercício
da palavra, foi artista das formas e das cores, como pintor, escultor e
desenhista. São, aliás, do próprio autor os desenhos que acompanham
esta edição.\footnote{Conforme a edição definitiva. A respeito da
relação desses desenhos com o texto, José Paulo Paes escreveu um ensaio
brilhante, que traça um novo caminho para a compreensão do romance:
``Sobre as ilustrações d'\textit{O Ateneu}''. In \textit{Gregos \&
Baianos} -- ensaios,  São Paulo, Brasiliense, 1985.} 
No colégio Abílio, editou um jornal manuscrito e
ilustrado, chamado \textit{Archote}. Como o
tom da folha era caricatural, e não poupava os professores e
funcionários, o menino, ainda com 11 anos, assinava sob um pseudônimo,
Fabricius. O exercício da charge foi a primeira manifestação do
espírito crítico, irônico e mordaz do escritor.


Interrompida aos 32 anos, a obra de Raul Pompéia pode
ser melhor interpretada em correspondência com sua postura participante
e interessada nos destinos da nação. Subjacente à ela revela"-se, para
além das inquietações do homem, a perspectiva de um plano reformador. A
escrita de \textit{O Ateneu} não se esquiva desse compromisso.

Raul Pompéia foi um prodígio. 


\section*{o pior dos mundos}

\epigraph{Até o micróbio, que é a morte, se educa e amansa.}{Raul Pompéia}

``Ateneu'' é a palavra com que os gregos antigos
designavam os edifícios dedicados a Atena, deusa da sabedoria, da razão
e da guerra. A partir do século \textsc{xviii}, várias instituições europeias de
ensino adotaram essa denominação. No Brasil, não foram poucos os
colégios de 2º grau com esse nome.


A leitura e as constantes referências ao romance permitem derivar do
substantivo Ateneu o verbo que indique um sistema de ensino.
``Ateneizar'' significaria homogeneizar por meio de um
procedimento didático segundo o qual um fala (ordena) e muitos ouvem
(obedecem). Ignorando o estímulo à criatividade e ao debate, no Ateneu
desconsideram"-se inclinações e potencialidades individuais. Em seu
lugar é instaurada a didática do controle e da uniformização das
diferenças.


O ensino, considerado desse modo, é fruto de uma
mentalidade eminentemente militar, típica do
2º Reinado, que confunde educação com formação moral. Essa característica se traduz,
por exemplo, na descrição de Bataillard, o professor de educação
física, enriquecida de metáforas militares: 

\begin{hedraquote}
Ao peito tilintavam"-lhe as agulhetas do comando, apenas de cordões
vermelhos em trança. Ele dava ordens fortemente, com uma vibração
penetrante de corneta que dominava à distância, e sorria à docilidade
mecânica dos rapazes. Como oficiais subalternos, auxiliavam"-no os
chefes de turma, postados devidamente com os pelotões, sacudindo à
manga distintivos de fita verde e canutilho.
\end{hedraquote}

Como num quartel, ou mesmo numa prisão, os alunos são
submetidos a uma rotina rígida e previsível, ao respeito cego às
hierarquias e às normas. Nesse ambiente, ``educar'' é procedimento
simples: consiste em punir e premiar aqueles que, respectivamente, não
se submetem e que se submetem ao sistema disciplinar. Em instituições
como essa, de violência surda e desconhecimento do outro, educar
significa adestrar.\footnote{ Sobre esse paralelo, aqui apenas
esboçado, vale a pena conhecer dois estudos capitais: o de M.~Foucault,
\textit{Vigiar e punir: nascimento da prisão}, Petrópolis, Vozes, 1999; 
e o de E.~Goffman, \textit{Manicômios, prisões e
conventos}, Perspectiva, São Paulo, 1987.}


Ao ingressar nesse sistema, a criança é deixada de lado, e submetida a
um processo de conversão em aluno: ``A mocidade ia transigindo do melhor
jeito com as bicudas imposições das circunstâncias''. Não raramente
traumática, a conversão depende de sua retirada brusca da esfera
familiar e da consequente superexposição à esfera pública. No espaço
escolar, espera"-se da criança um desempenho e uma conduta, que serão
testados e avaliados por repetidos mecanismos de seleção, vigilância e
enquadramento. Num Ateneu, a obediência é sempre fruto do temor:

\begin{hedraquote}
A sala geral do estudo tinha inúmeras portas. Aristarco fazia aparições,
de súbito, a qualquer das portas, nos momentos em que menos se podia
contar com ele. 
Levava as aparições às aulas, surpreendendo professores e discípulos.
Por meio deste processo de vigilância de inopinados, mantinha no
estabelecimento por toda a parte o risco perpétuo do flagrante como uma
atmosfera de susto. Fazia mais com isso que a espionagem de todos os
bedéis. Chegava o capricho a ponto de deixar algumas janelas ou portas
como vetadas a fechamento para sempre, com o fim único de um belo dia
abri"-las bruscamente sobre qualquer maquinação clandestina da
vadiagem. Sorria então no íntimo, do efeito pavoroso das armadilhas, e
cofiava os majestosos bigodes brancos de marechal, pausadamente, como
lambe o jaguar ao focinho a pregustação de um repasto de sangue.  
\end{hedraquote}

Algo que a todo momento fica claro nesse novo espaço ---
no romance, designado pela sala de aula, o pátio e a piscina --- é que
ele não é um meio de \textit{transformação},
mas de \textit{preparação} para a sociedade.
Nesse ambiente corrompido e corruptor, os alunos passam naturalmente de
oprimidos a opressores. Incapazes de se voltar contra o sistema que os
controla e os mantém em estado de inação intelectual, reproduzem entre
si as mesmas relações de dominação às quais estão submetidos. Não
raramente, esses episódios são evocados pelo narrador num ritmo febril,
com sintaxe desarmada, liberando o fluxo de uma consciência atormentada
pela experiência:

\begin{hedraquote}
Não me enganavam mais os pequeninos patifes. Eram infantis, alegres,
francos, bons, imaculados, saudade inefável dos primeiros anos, tempos
da escola que não voltam mais!\ldots{} E mentiam todos!\ldots{} Cada rosto amável
daquela infância era a máscara de uma falsidade, o prospecto de uma
traição. Vestia"-se ali de pureza a malícia corruptora, a ambição
grosseira, a intriga, a bajulação, a covardia, a inveja, a sensualidade
brejeira das caricaturas eróticas, a desconfiança selvagem da
incapacidade, a emulação deprimida do despeito, a impotência, o
colégio, barbaria de humanidade incipiente, sob o fetichismo do Mestre,
confederação de instintos em evidência, paixões, fraquezas, vergonhas,
que a sociedade exagera e complica em proporção de escala, respeitando
o tipo embrionário, caracterizando a hora presente, tão desagradável
para nós, que só vemos azul o passado, porque é ilusão e distância.
\end{hedraquote}

A escola cumpre, assim, o papel de laboratório social, que expõe o aluno
às provas da desconfiança, do vexame e do desprezo para dá"-lo como
``formado'', isto é, rendido pela sociedade.

Se essa rendição é conquistada à força das
circunstâncias e sob um regime disciplinar restrito, também é resultado
de um desvio, de um processo de falseamento da verdade.\footnote{Alfredo Bosi, 
valendo"-se de uma analogia com as águas turvas da
piscina do colégio (cenário de um de seus quadros mais impactantes),
refere"-se a essa fuga da verdade com a expressão ``pedagogia viscosa'',
que se opõe à pedagogia transparente, através da qual as intenções
seriam condizentes com os métodos. ``\textit{O Ateneu}, opacidade e
destruição''. In \textit{Céu, inferno -- ensaios de crítica literária e
ideológica}, Duas Cidades/Ed.~34, São Paulo, 2003, p.~57.} 
A excitação dos sentidos se obtém por meio de
rituais espetaculares, como a ``festa da educação física'', em que o
colégio se transforma num enorme cartão de visitas para os pais e
alunos ingressantes. O colégio é uma realidade cenográfica, um mundo
premeditado, calculado e representado na imagem de diretor:

\begin{hedraquote}
Uma hora trovejou"-lhe à boca, em sanguínea eloquência,
o gênio do anúncio. Miramo"-lo na inteira expansão oral, como, por
ocasião das festas, na plenitude da sua vivacidade prática.
Contemplávamos (eu com aterrado espanto) distendido em grandeza épica ---
o \textit{homem sanduíche} da educação
nacional, lardeado entre dois monstruosos cartazes. Às costas o seu
passado incalculável de trabalhos; sobre o ventre, para a frente, o seu
futuro: a \textit{réclame} dos imortais projetos.
\end{hedraquote}

O pedagogo possui ali duas habilidades: a retórica falaciosa e a
teatralidade. Estas, ao invés de chamarem a atenção para determinado
assunto e estimularem a reflexão a seu respeito, são antes finalidades
em si mesmas. A didática não é um meio de transmissão ou de produção de
conhecimento, mas um truque para prender a atenção. A finalidade da
escola é atrair e manter os alunos. Por isso ela é perita em ostentar
uma posição, em vender uma imagem:

\begin{hedraquote}
A irradiação do
\textit{réclame} alongava de tal modo os
tentáculos através do país, que não havia família de dinheiro,
enriquecida pela setentrional borracha ou pela charqueada do sul, que
não reputasse um compromisso de honra com a posteridade doméstica
mandar dentre seus jovens, um, dois, três representantes abeberar"-se à
fonte espiritual do \textit{Ateneu}.
\end{hedraquote}

Bem"-sucedido como empresa e conduzido pela ``ética'' do
\textit{marketing}, o Ateneu jamais protesta
contra a sociedade, ele é sempre similar à ela --- à biologia dos mais
adaptados, à gramática inflexível das linhas de montagem e à aritmética
incondicional da corrida pelo lucro. 

\begin{hedraquote}
E não se diga que é um viveiro de maus germens,
seminário nefasto de maus princípios, que hão de arborescer depois. A
corrupção que ali viceja, vai de fora. Os caracteres que ali triunfam,
trazem ao entrar o passaporte do sucesso, como os que se perdem, a
marca da condenação.
\end{hedraquote}

E acrescente"-se ainda: 

\begin{hedraquote}
Ensaiados no microcosmo do internato, não há mais surpresas no
grande mundo lá fora, onde se vão sofrer todas as convivências,
respirar todos os ambientes; onde a razão da maior força é a dialética
geral, e nos envolvem as evoluções de tudo que rasteja e tudo que
morde, porque a perfídia terra"-terra é um dos processos mais eficazes
da vulgaridade vencedora; onde o aviltamento é quase sempre a condição
do êxito, como se houvesse ascensões para baixo; onde o poder é uma
redoma de chumbo sobre as aspirações altivas; onde a cidade é franca
para as dissoluções babilônicas do instinto; onde o que é nulo, flutua
e aparece, como no mar as pérolas imersas são ignoradas e sobrenadam ao
dia as algas mortas e a espuma.
\end{hedraquote}

Formados nos Ateneus, de ontem e de hoje, integram"-se anualmente à
sociedade enormes contingentes de massa muda, com uniformidade de
opinião, incapazes de se revoltar espontaneamente apenas por ocuparem
os dois únicos postos à disposição: de opressores ou de oprimidos. Não
há mais perigo em lhes conceder liberdade intelectual. Ao aceitarem
essas posições como desígnios transcendentais ou de seleção, a
estupidez lhes cabe como proteção. São como a formiga, que pode ver
pequenos objetos, mas não enxerga os grandes. O mundo se moderniza, o
curso da história se mantém.

Numa época em que a desigualdade entre os homens é aceita como o preço
da civilização, as escolas e faculdades que se especializaram em
formar, não seres críticos, mas, como se costuma dizer, ``alunos bem
preparados'', podem ter se tornado o pior dos mundos.


\section*{romance de formação}

\epigraph{Devolva-me a juventude.}{\textit{Fausto}, Goethe}

A literatura brasileira é sempre muito pensada em termos
de escolas e movimentos literários.\footnote{ No final do volume, o
leitor encontrará uma coletânea que, ao retraçar a evolução na recepção
crítica do romance, confirma a presente afirmação.} É
natural que, ao se aproximar de um romance tão rotulado, como foi
\textit{O Ateneu}, o crítico se sinta
inclinado a assumir uma posição no cenário criado. Não é este o caso.
Proponho aqui um deslocamento do contexto de recepção da obra.

``Romance de formação'' é a tradução mais corrente da
principal tradição narrativa em língua alemã.\footnote{Em inglês,
pesquisar por \textit{coming"-of"-age novel} ou
\textit{apprenticeship novel}.} Seu surgimento está
vinculado ao caldo cultural iluminista, responsável pela ruptura com a
mentalidade e as doutrinas metafísicas e fatalistas da Igreja católica.


No século \textsc{xviii}, a valorização da razão como instrumento de renovação
abalava os mecanismos tradicionais de manutenção das ordens política e
social, e refletia a ascensão da burguesia, uma classe sem amarras com
a Igreja. Autores como Hume, Goethe, Lessing, Voltaire, Rousseau,
Locke, Diderot e Montesquieu, a despeito de suas diferenças,
consolidavam uma visão de mundo menos espiritualista e mais cética,
empírica e materialista.


O \textit{Bildungsroman} surge nesse contexto de transformações sociais e produção cultural como
representação da posição e do papel mais central e autônomo do homem no
mundo. \textit{Emílio} (1762), o romance
pedagógico de Rousseau a respeito da educação do protagonista desde seu
nascimento até os vinte anos de idade, é talvez o texto que mais
nitidamente abra caminho para o ``romance de formação'', cujo paradigma é
tradicionalmente considerado \textit{Os anos de aprendizagem
de Wilhelm Meister} (1795--96), de Goethe.


A característica predominante do
\textit{Bildungsroman} é o tema de que
trata: o desenvolvimento interior de um protagonista criança ou jovem
mediante suas dificuldades de interação com os demais e o meio em que vive. 

Essa definição se enriquece da constatação de que em nenhum século
anterior ao \textsc{xix} a literatura privilegiou tanto o protagonista em
formação. A tabela abaixo situa em ordem cronológica de aparição alguns
desses célebres jovens heróis do romance daquele século.

\begin{footnotesize}
\begin{center}
\begin{tabular}{lll}
Herói             & Autor 		& Obra \\\hline
Elizabeth Bennet  & Jane Austin  	& \textit{Orgulho e preconceito} (1813)\\
Ema               & Jane Austin 	& \textit{Ema} (1816)\\
Julien Sorel      & Stendhal 		& \textit{O vermelho e o negro} (1830) \\
Louis Lambert     & Balzac 		& \textit{Louis Lambert} (1832)\\
Eugene Onegin     & Pushkin 		& \textit{Eugene Onegin} (1833) \\
Eugène Rastignac  & Balzac 		& \textit{O pai Goriot} (1834) \\
Oliver Twist      & Dickens 		& \textit{Oliver Twist} (1838) \\
Renzo Tramaglino  & Manzoni 		& \textit{Os noivos} (1842) \\
Lucien de Rubempré& Balzac 		& \textit{As ilusões perdidas} (1843) \\
David Copperfield & Dickens 		& \textit{David Copperfield} (1849) \\
Bazarov           & Turguêniev	 	& \textit{Pais e filhos} (1861) \\
Raskólnikov       & Dostoiévski 	& \textit{Crime e castigo} (1866) \\
Frédéric Moreau   & Flaubert 		& \textit{Educação sentimental} (1869) \\ 
Dorothea Brook    & George Eliot 	& \textit{Middlemarch} (1871) \\ 
Pinóquio          & Carlo Collodi 	& \textit{Pinóquio} (1883) \\
Georges Duroy     & Maupassant	 	& \textit{Bel-Ami} (1885) \\
Fabrizio del Dongo& Stendhal 		& \textit{A cartucha de Parma} (1889) \\
Jude Fawley       & Thomas Hardy 	& \textit{Judas o obscuro} (1895) 
\end{tabular}
\end{center}
\end{footnotesize}

Esta seleção, realizada ao sabor da memória e um pouco
ao acaso das consultas, poderia ser multiplicada, mas é o bastante para
oferecer um diagnóstico significativo a respeito da predileção pelo
jovem protagonista no romance do século \textsc{xix}. Sua escolha não foi
casual. A juventude, enquanto símbolo de uma época, parece atender a um
anseio próprio da passagem para a modernidade. Ela personifica a
instabilidade, seja da sociedade capitalista e industrial, seja da
personalidade individual; o dinamismo das classes, e dos valores; a
brevidade dos sistemas, e das certezas; a mentalidade, enfim, que busca
explicações não mais no passado, mas no futuro. No
\textit{Fausto}, de Goethe, a juventude é o
primeiro presente que Mephisto oferece ao protagonista.\footnote{ A
respeito da relação entre o \textit{Bildungsroman} e o romance europeu
do século \textsc{xix}, e da eleição do jovem como figura simbólica da
modernidade, conferir o estudo de Franco Moretti, \textit{The way of
the world -- the Bildungsroman in European Culture} (trad.), Verso,
London, 1987.} 

O \textit{Bildungsroman} tem
como necessidade constitutiva essa eleição do jovem protagonista em
atrito com o mundo. Um de seus exemplos tradicionais é
\textit{David Copperfield} (1849), a obra"-prima de Charles Dickens. 
O romance, considerado o mais
autobiográfico de seu autor, traduz"-se num contundente ataque às
instituições públicas inglesas. Sua estrutura ergue"-se em torno do
processo de formação do protagonista, desde a infância difícil,
passando pelo aprendizado com um advogado, até o êxito literário. 


O processo de amadurecimento pressupõe, em enredos de
natureza tão similar, um protagonista em constante choque com a
realidade exterior. Ao mesmo tempo, requer que seu desconforto não dê
lugar à apatia, mas que seja combatido através da incorporação das
experiências vividas. Por isso, o
\textit{Bildungsroman} não entrega ao leitor
uma imagem pré"-estabelecida do herói; é antes o processo de formação
de sua personalidade. Os acontecimentos não movimentam simplesmente a
personagem na cena, alterando seu destino ou sua condição de vida. Quem
é alterada é a própria personagem. A dinâmica do romance é, portanto, a
dinâmica de uma personalidade.\footnote{Entre os trabalhos a esse respeito, 
é de suma importância o estudo de Mikhail Bakhtin, O romance de educação na história do
realismo, in \textit{Estética da criação verbal}, Martins Fontes, São Paulo, 2000.} 

O ``romance de formação'', por vezes referido como
``romance de educação'' ou ``pedagógico'', privilegia o espaço da escola,
considerado como um ambiente avesso ao bem"-estar do aluno. É o caso
de \textit{O jovem Törless} (1906), que,
baseado nas experiências escolares de seu autor, Robert Musil, narra o
desenvolvimento de um adolescente em um internato.

O gênero passa por desdobramentos sucessivos, incorporando forte crítica
ao sem"-sentido da vida burguesa e acentuando"-lhe o grau de
questionamento moral e o clima de ceticismo, ansiedade e angústia. 

Em termos estilísticos, \textit{O retrato do
artista quando jovem} (1916), de James Joyce, é uma
narrativa nitidamente moderna, isto é, auto"-consciente de seus
procedimentos de escrita, e revela a evolução do
\textit{Bildungsroman}. Também considerado o
romance autobiográfico de Joyce, narra episódios traumáticos referentes
à sua infância, tal como o castigo de palmatória, que o jovem
protagonista sofre de um bedel inábil, desconfiado de que o menino (que
havia perdido os óculos) não fazia sua tarefa em sala por mera
preguiça. O ensino jesuíta é o responsável pela impressionante cultura
clássica, pelo conhecimento de línguas, mas também pelos fantasmas de
Joyce. No romance, Stephen Dedalus é um autor"-menino que procura
libertar"-se da educação recebida e dos dilemas da pátria para
encontrar seu próprio caminho como
escritor.\footnote{Alguns pesquisadores conferem uma abrangência maior ao
\textit{Bildungsroman}, forçando seus traços mais característicos. A consideração de uma
protagonista feminina nesse tipo de narrativa é uma dessas inovações, e
possibilita, a despeito das acomodações necessárias para tanto, a
inclusão, por exemplo, de \textit{Perto do coração selvagem} (1944), de Clarice
Lispector como uma variação do \textit{Bildungsroman}.
Lembremos que, neste romance, a fabulação é mínima. A luta de Joana, de
menina a adulta, ao lado do amante anônimo, é, antes de tudo, uma luta
com a linguagem. Encontrar"-se significa, não descobrir seu papel na
sociedade, adaptar"-se a ela ou enxergar um destino para suas
inquietações, mas encontrar um meio próprio de auto"-expressão. Outro
livro lembrado, de Clarice Lispector, é \textit{Uma aprendizagem ou o livro dos
prazeres}. Para um estudo pormenorizado a esse respeito, consultar, de Cristina Ferreira Pinto,
\textit{O Bildungsroman feminino: quatro exemplos brasileiros}, Perspectiva, São Paulo, 
1990.  \\No século \textsc{xx}, são narrativas muito lembradas
\textit{Os Buddenbrooks} (1901), de Thomas Mann, e \textit{O tambor de
lata} (1959), de Günter Grass. Mais recentemente, vale citar o romance
de Edward Said, \textit{Out of place} (2002). Na literatura brasileira,
podemos incluir, um tanto arbitrariamente, nessa pequena coletânea:
``Conto de escola'', de Machado de Assis; \textit{Doidinho}, de José Lins
do Rego; \textit{Mundos Novos}, de Otávio de Faria; \textit{Limite
branco}, de Caio Fernando Abreu; e \textit{Em nome do desejo}, de João Silvério Trevisan.} 

Na escola ou fora dela, o protagonista luta, mais especificamente, para
satisfazer sua demanda de autoconhecimento. 
Conhecer"-se implica, nesse sentido, não um
ato de separação e isolamento, mas uma tentativa de conciliação, de
compreensão profunda da realidade histórica. 

É preciso atentar para esse aspecto do ``romance de
formação'', porque o costume de identificá"-lo simplesmente como
``romance subjetivo'' ou ``introspectivo'' apaga justamente seu traço mais
distintivo. No \textit{Bildungsroman}, a
realidade histórica tem uma ação transformadora sobre o indivíduo.
Aquilo que é casual e cotidiano, uma vez integrado às demais
experiências, e ao repertório afetivo e intelectual do protagonista,
torna"-se essencial. Seu vetor é, portanto, a harmonia, e reside aí
seu caráter híbrido. 

Expliquemos melhor. Se o autoconhecimento define"-se
pela interação do indivíduo com a sociedade, o caráter subjetivo do
texto não apaga sua outra dimensão, objetiva. Realidade íntima e
histórica devem manter"-se em contato. Daí uma narrativa difícil de
definir: que nem é idealista (porque não transfigura a realidade
histórica), nem realista (porque o tempo histórico é sempre filtrado
pelo tempo psicológico). Por outro ângulo, trata"-se de uma narrativa
que se aproxima da \textit{autobiografia} (na focalização tanto externa 
quanto interna do protagonista, isto é,
no desenrolar das experiências vividas e incorporadas), mas que dela
também se afasta (pela prática livre da abstração e da conjectura, que
leva à universalização de uma história, tomada apenas a princípio como
individual).

Mas esse raciocínio também esclarece algumas especificidades do gênero,
de que se pode extrair decorrências mais conclusivas.

Se a realidade histórica é filtrada pela realidade psicológica, a
condução da narrativa tende a ser fragmentada, sem princípio rígido de
unidade ou encadeamento lógico entre os episódios. Em alguns romances
desse tipo é possível, inclusive, inverter a ordem de alguns capítulos
sem grande prejuízo para a trama. O leitor fica com a sensação nítida
de que a personagem não é conduzida pela narrativa, e sim que é esta
que se encadeia em função do protagonista.

Também é possível afirmar que a temática da aprendizagem, da formação, é
fator impeditivo para a morte do herói no final do romance,
lugar"-comum para o protagonista em qualquer gênero narrativo. Ele
sobrevive. Mas seu destino não é selado: a aprendizagem é um processo
doloroso e contínuo --- mais condizente, portanto, com um enredo que
permaneça em aberto, refém da impossibilidade de ajuste entre o eu e o
mundo.
\asterisc

\textit{O Ateneu} é o principal ``romance de
formação'' brasileiro.

Há duas instâncias narrativas cujo tratamento é singular no romance: o
foco narrativo e o tempo.

Sérgio, já adulto, narra a estória de sua infância, mais precisamente do
período de dois anos, num colégio interno. No romance, há a presença
marcante de pontos de vista diferentes: do Sérgio adulto, que narra
através da memória; e do Sérgio menino, herói, que viveu as
experiências narradas. A narração é uma instância em constante
transição entre essas duas vozes. Através da superposição de
perspectivas, o narrador transita da terceira para a primeira pessoa,
do distanciamento para a participação na narrativa. Mas mesmo narrando
da terceira pessoa, dá preferência sistemática à focalização interna
das personagens. O narrador identifica"-se, portanto, com seu
protagonista, conhece suas vontades e hesitações, e assume a
perspectiva da própria personagem. Essa proximidade sugere uma leitura
autobiográfica do romance. Em outros termos, a identificação
narrador"-protagonista espelharia a identificação autor"-narrador, o
que acresce à leitura uma curiosidade menos estética do que antropológica. 

A onisciência do narrador é favorecida assim por dois pontos de vista: o
da criança, resgatado pela memória, e o do adulto, no presente da
narrativa. Este, ao relembrar sua infância, oscila entre as
consequências imediatas dos acontecimentos e suas decorrências a longo
prazo. Há, portanto, dois presentes: o dos acontecimentos e o da narrativa.

O romance enfoca a educação intelectual, a integração social e a
afirmação sexual de Sérgio e de seus colegas: o dissimulado Sanches; o
injustiçado e vingativo Franco; o atlético Bento
Alves; e o enamorado Egberto. Entre eles, o sentimento de amizade é
maculado pelos impulsos primitivos. Daí o tom da narração ser o de um
ressentimento constante. A narrativa foca também as relações dos alunos
com Aristarco, o diretor; e de Sérgio com Ema, a esposa do diretor.

Os três processos de formação são percursos penosos e sempre mal
resolvidos. O aprendizado das matérias é matizado por um ensino
retrógrado, que procede pelo falseamento da verdade, e por uma relação
impositiva entre professor e aluno. O convívio entre os colegas é
marcado pela disputa, pela malícia, e pela lei do mais forte. A
descoberta sexual resulta de um jogo de posse, em que o outro é
encarado como meio de se conquistar algo ou de satisfazer algum desejo.

Importa ao narrador conhecer o real caráter e as intenções dos colegas e
professores por trás das aparências, a casca grossa das regras de
conduta, da retórica e das formalidades. Mas, mais importante do que
conhecer o outro, é valer"-se dessa interação para conhecer"-se a si
mesmo. No chão social sobre o qual se vive no Ateneu, Sérgio nunca está
em repouso, mas em constante processo de adaptação. Por isso, o
realismo atmosférico do romance toca incessantemente a esfera mais
íntima do psicológico, e esta se abrirá, como veremos mais a frente,
para uma interpretação universalizante da narrativa. 

A distância temporal e a manutenção da primeira pessoa possibilitam ao
narrador narrar a própria subjetividade (infantil), objetivamente
(madura). Assim, notemos: não é o menino Sérgio que fala de si, mas o
narrador que torna linguisticamente maduro os sentimentos da
personagem, que ela mesma não poderia expressar. Não é Sérgio que vê ou
sente, este seria muito mais raso e confuso se pudesse se apresentar a
si mesmo, mas é ele próprio visto como alguém que vê e que sente, e
através disso é que é julgado. Veja"-se, por exemplo, como a voz do
narrador contribui decisivamente para os efeitos de sentido do romance,
sobretudo no que concerne às comparações e metáforas de que se vale com
grande recorrência: 

\begin{hedraquote}
Este foi o caráter que mantive, depois de tão várias oscilações. Porque
parece que às fisionomias do caráter chegamos por tentativas,
semelhante a um estatuário que amoldasse a carne no próprio rosto,
segundo a plástica de um ideal; ou porque a individualidade moral a
manifestar"-se, ensaia primeiro o vestuário no sortimento psicológico
das manifestações possíveis. 
\end{hedraquote}

Este filtro narrativo, constituído de distanciamento e maturidade, é o
que garante ao romance as identificações com o impressionismo e com o
romance psicológico. No plano discursivo, sua consequência é a inserção
de comentários na narrativa que alteram seu andamento, apontando para o
futuro: o narrador antecipa as consequências, que recairiam sobre si ou
sobre outra personagem, de determinada experiência narrada.
Acompanhemos a abertura do capítulo \textsc{iii}:

\begin{hedraquote}
Se em pequeno, movido por um vislumbre de luminosa
prudência, enquanto aplicavam"-se os outros à peteca, eu me houvesse
entregado ao manso labor de fabricar documentos autobiográficos, para a
oportuna confecção de mais uma \textit{infância
célebre}, certo não registraria, entre os meus
episódios de predestinado, o caso banal da natação, de consequências,
entretanto, para mim, e origem de dissabores como jamais encontrei tão amargos.
\end{hedraquote}

O romance está repleto desse tipo de construção. Veja"-se, por exemplo,
o efeito de sentido produzido pela narração da abordagem que Sanches
faz ao narrador, de sugestão erótica, e que é calcinada pela menção à
futura profissão da personagem, intensificada pela repetição e pelo
adjetivo ``grave'':

\begin{hedraquote}
Como não aprendi a pergunta, o Sanches repetiu. Escapou"-me
involuntário o riso\ldots{} Abarbava"-me a mais rara espécie de
pretendente! Eu ria com franqueza, mas abismado. Era de uma
extravagância original aquele Sanches! Hoje, ele é engenheiro em uma
estrada de ferro do sul, um grave engenheiro\ldots{} 
\end{hedraquote}

O contraste exemplifica no livro seu caráter formador, de rito de
passagem.

O futuro do pretérito é o tempo"-chave da narrativa. As antecipações
que possibilita são em grande parte responsáveis pela qualidade
diferenciada do romance. O narrador é então capaz de avaliar, pesar e
julgar suas experiências infantis. Seu critério de seleção é a
decorrência moral dessas experiências, e é por essa ótica que as
situações de convívio são preferidas às de sala de aula. 

A escola é um espaço privilegiado no ``romance de
formação''. Até meados do século passado, era comum o regime de
internato, que significava o abandono do espaço familiar e,
consequentemente, a imposição da necessidade de adaptação imediata ao
novo meio. Em \textit{O Ateneu}, cujo
\textit{incipti} (o início) está entre os
mais célebres da literatura brasileira, as palavras ``mundo'', ``coragem''
e ``luta'' sintetizam exemplarmente a transformação resultante do
ingresso nesse novo espaço. Neste trecho, como, de resto, na maior
parte das passagens mais significativas do romance, nota"-se o mesmo
uso temporal apontado acima:

\begin{hedraquote}
``Vais encontrar o mundo, disse"-me meu pai, à porta do
\textit{Ateneu}. Coragem para a luta.''
Bastante experimentei depois a verdade deste aviso, que me despia, num
gesto, das ilusões de criança educada exoticamente na estufa de carinho
que é o regime do amor doméstico, diferente do que se encontra fora,
tão diferente, que parece o poema dos cuidados maternos um artifício
sentimental, com a vantagem única de fazer mais sensível a criatura à
impressão rude do primeiro ensinamento, têmpera brusca da vitalidade na
influência de um novo clima rigoroso.
\end{hedraquote}

O Ateneu é um \textit{pathos}
hostil. Ingressar ali significa despir"-se das ilusões de criança. O
subtítulo do romance, ``crônica de saudades'', revela, portanto, o tom
irônico da narrativa: ``Eufemismo, os felizes tempos''. A saudade é
``hipócrita''. O processo de recordação é antes um meio que o narrador
encontra para livrar"-se do fel do passado; algo que fica a todo
momento aparente em sua semântica do horror, composta por termos e
expressões que são próprios da guerra, da disputa. Para este narrador,
o ingresso no colégio marca a despedida definitiva do seio familiar,
assumindo o sentido mais amplo de um ritual de passagem.\footnote{ De
modo similar ao encontrado no romance brasileiro, Joyce reforça esse
sentido num dos trechos iniciais de \textit{O retrato do artista quando
jovem}: ``Sua mãe dissera"-lhe para não falar com meninos grosseiros no
colégio. Aquilo é que era mãe! No primeiro dia, no castelo, ao se
despedir dele, ela tinha erguido o véu, dobrando"-o por cima do nariz,
para poder beijá"-lo; e tanto o nariz como os olhos dela estavam
vermelhos. Mas fingira não perceber que ela estava a ponto de chorar. E
o pai então lhe dera duas moedas de 5 xelins para ele ficar com
dinheiro miúdo no bolso. E o pai lhe dissera que se precisasse de
qualquer coisa que escrevesse para casa e que nunca, fizessem"-lhe lá
o que fosse, desse parte de qualquer colega. Depois, à porta do
castelo, o reitor estendera a mão a seu pai e a sua mãe, enquanto a
sotaina dele esvoaçava na brisa; e o carro tinha ido embora, levando
seu pai e sua mãe. Lá do carro eles o tinham chamado alto, agitando as
mãos: `Adeus, Stephen, adeus! Adeus, Stephen, adeus!'''. In James
Joyce, \textit{Retrato do artista quando jovem}, trad. de José Geraldo
Vieira, São Paulo, Abril, 1971, pp.11--12.}

Nesse espaço hostil, a condução dos acontecimentos é
feita ao sabor da memória do narrador. Não há uma linha narrativa bem
demarcada no romance. A estruturação da trama é bastante frágil, sem um
rígido encadeamento lógico. 
O que menos importa é o que veio antes e o que veio depois. O romance é composto
por quadros que mantêm certa independência entre si. Por vezes,
a frequência com que as lembranças se dão acelera a narrativa a partir
de saltos temporais, como no início do capítulo \textsc{viii}, ``No ano seguinte,
o \textit{Ateneu} revelou"-se"-me noutro aspecto''.

Esses quadros, referentes à realidade escolar, correspondem à realidade
política e social. As relações hierarquizadas e dominadas pela força
têm o vetor apontado para a monarquia, contra a qual o romance se
volta, fazendo da escola uma alegoria do regime. Não por acaso
frequentam"-na a princesa Isabel e o próprio imperador. Na ``festa da
educação física'', ocasião solene em que o colégio abre as portas para a
corte, o arrivismo político de Pompéia manifesta"-se na postura do
filho de Aristarco:

\begin{hedraquote}
Uma coisa o entristeceu, um pequenino escândalo. Seu filho Jorge, na
distribuição dos prêmios, recusara"-se a beijar a mão da princesa,
como faziam todos ao receber a medalha. Era republicano o pirralho!
Tinha já aos quinze anos as convicções ossificadas na espinha
inflexível do caráter! Ninguém mostrou perceber a bravura. Aristarco,
porém, chamou o menino à parte. Encarou"-o silenciosamente e --- nada
mais. E ninguém mais viu o republicano! Consumira"-se naturalmente o
infeliz, cremado ao fogo daquele olhar! Nesse momento as bandas tocavam
o hino da monarquia jurada, última verba do programa. 
\end{hedraquote}

No colégio, o correspondente direto ao imperador é
Aristarco. Reforçam este caminho interpretativo os traços, por vezes
carregados, de mais aguda sátira com que o narrador pinta a figura
burguesa, retrógrada e, por vezes, ambiguamente perversa do diretor. Em
suas próprias palavras, revela"-se, sob os mantos transparentes de um
falso moralismo conservador, a figura do
\textit{voyeur}, a marca do velho lobo que,
sob o pretexto de corrigir os males do instinto, alimenta
ofegantemente, como uma besta sedenta, seu
\textit{fetiche}, fonte indisfarçável de
prazer: 

\begin{hedraquote}
``Um trabalho insano! Moderar, animar, corrigir esta massa de caracteres,
onde começa a ferver o fermento das inclinações; encontrar e encaminhar
a natureza na época dos violentos ímpetos; amordaçar excessivos
ardores; retemperar o ânimo dos que se dão por vencidos precocemente;
espreitar, adivinhar os temperamentos; prevenir a corrupção; desiludir
as aparências sedutoras do mal; aproveitar os alvoroços do sangue para
os nobres ensinamentos; prevenir a depravação dos inocentes; espiar os
sítios obscuros; fiscalizar as amizades; desconfiar das hipocrisias;
ser amoroso, ser correto; proceder com segurança, para depois duvidar;
punir para pedir perdão depois\ldots{} Um labor ingrato, titânico, que
extenua a alma, que nos deixa acabrunhados ao anoitecer de hoje, para
recomeçar com o dia de amanhã\ldots{} Ah! meus amigos, conclui ofegante, não
é o espírito que me custa, não é o estudo dos rapazes a minha
preocupação\ldots{} é o caráter! Não é a preguiça o inimigo, é a
imoralidade!'' Aristarco tinha para a palavra uma entonação especial,
comprimida e terrível, que nunca mais esquece quem a ouviu dos seus
lábios. ``A imoralidade!''
\end{hedraquote}

Vem a propósito considerar a escolha de Pompéia pelo nome ``Aristarco'',
segundo a etimologia: ``Arist"-'' (presente em ``aristocrata'', é o
superlativo de ``bom'') e ``arc"-'' (presente em ``monarca'', significa
``governar''); portanto, Aristarco significa ``o melhor governante''.

No final do romance, o incêndio que destrói o colégio é de autoria de
Américo, menino robusto, de origem rural, visto pelos colegas como ``uma
fera respeitável''. Por contraste, o único aluno que morre é o doente e
raquítico Franco. Esse final suscita duas perguntas: por que o
incêndio, e por que sua autoria dada a Américo, um personagem
coadjuvante, e não a Sérgio?

Por um lado, o final apoteótico é tipicamente folhetinesco. Os romances
deviam vender jornais, e o final arrebatador era um lugar"-comum
eficiente para tanto, a exemplo do que ocorre ainda hoje nas
telenovelas. É possível que Pompéia tivesse feito uma concessão,
portanto. Mas mais interessante que isso é considerar que esse final
tem decorrências que apontam para uma dimensão alegórica no contexto em
que é publicado.

Estamos em 1888, às vésperas da proclamação da
República, e num momento de larga emancipação da América Latina de suas
amarras colonialistas com o Velho Mundo. Nossos vizinhos conquistaram
sua independência e passam por um período de afirmação política e
cultural, com a promulgação de constituições e tentativas de
reorganização social. No Brasil, como vimos, o envolvimento de Pompéia
com o abolicionismo e o republicanismo é integral, mesmo visceral. A
apenas um ano da proclamação da República, Pompéia ateia fogo no
Ateneu, e a destruição em si consiste numa catarse libertadora da
tirania. A primorosa descrição da cena final aponta para uma múltipla
decadência: 1) a decadência da aristocracia e do império, simbolizados
na figura de seu imperador, Aristarco, ironicamente descrito como
sobrevivente ``imóvel'' e ``desolado'' ao lado dos escombros do colégio: 

\begin{hedraquote}
O Ateneu devastado! O seu trabalho perdido, a conquista inapreciável dos
seus esforços!\ldots{} Em paz!\ldots{} Não era um homem aquilo; era um
\textit{de profundis}. [\ldots{}] Ele, como um
deus caipora, triste, sobre o desastre universal de sua obra.
\end{hedraquote}

2) A decadência de um sistema de ensino, e com ele de um sistema de crenças
e valores, enfim, de uma visão de mundo, simbolizada pelos globos,
terrestres e celestes, esfolados e rachados: 

\begin{hedraquote}
Lá estava: em roda amontoavam"-se figuras torradas de geometria,
aparelhos de cosmografia partidos, enormes cartas murais em tiras,
queimadas, enxovalhadas, vísceras dispersas das lições de anatomia,
gravuras quebradas da história santa em quadros, cronologias da
história pátria, ilustrações zoológicas, preceitos morais pelo
ladrilho, como ensinamentos perdidos, esferas terrestres contundidas,
esferas celestes rachadas; borra, chamusco por cima de tudo: despojos
negros da vida, da história, da crença tradicional, da vegetação de
outro tempo, lascas de continentes calcinados, planetas exorbitados de
uma astronomia morta, sóis de ouro destronados e incinerados\ldots{}
\end{hedraquote}


Note"-se a frase final do período: ``sóis de ouro destronados e
incinerados''. A associação entre os termos ``sóis'' e ``destronados'' é
inequívoca: há aqui uma referência ao ``Rei Sol'', cognome de Luís \textsc{xiv},
rei da França durante toda a segunda metade do século \textsc{xvii} e início do
\textsc{xviii}, construtor do palácio de Versalhes, e a quem é atribuída a
famosa frase ``\textit{L'État c'est moi}'', ``O Estado sou eu'', um dos símbolos
maiores da monarquia. No romance, tudo isso é pateticamente reduzido a chamas,
e ao cadáver de um menino de quatorze anos, raquítico e problemático,
chamado Franco. Assim, é mais coerente que não seja Sérgio, mas
Américo, o autor do incêndio. Américo que, lembremos, era novidade:
aluno recém"-matriculado no colégio, e que de lá tentara fugir. Eis o
contraponto entre o Velho e Novo Mundo.

Estamos novamente no território do símbolo. A bem dizer,
nos dois anos que Sérgio passa no Ateneu, não se pode afirmar que há
efetivamente uma transformação no seu modo de ver o mundo, porque pouco
temos acesso a isso. É certo que ele tem de se adaptar ao novo
ambiente, algumas mudanças de atitude são perceptíveis, mas quem deixa
o Ateneu é ainda uma criança. Por que, então, ``romance de formação''? O
caráter original do romance está justamente no modo como se realizou
seu traço memorialista, que mantém Sérgio como o narrador adulto de uma
fase de sua infância. É nessa justaposição narrativa que se surpreende
a transformação ocorrida, entre o menino e o adulto, entre as
experiências vividas e as experiências narradas. A leitura mais
interessante do romance não se prende ao enredo, porque não assistimos
ao menino crescendo e se transformando em homem, e sim à exploração de
seu foco narrativo, que aponta para os episódios que seriam essenciais
nessa transformação. O que é simbólico nisso é que ela não ocorre
apenas na instância individual, mas na coletiva, porque a dimensão dada
ao incêndio do colégio sela a transição de dois mundos, abrindo caminho
para a construção, a \textit{formação} de
uma nova nação, e mesmo de um novo continente (implicado no nome
próprio Américo). É, afinal, da utopia de um Brasil republicano e livre --- ao
mesmo tempo cara e fatal para Pompéia ---, que um Sérgio adulto, já
formado portanto, narra e julga com maturidade suas desventuras
infantis num velho mundo chamado Ateneu. 
