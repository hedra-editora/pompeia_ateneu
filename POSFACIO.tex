\chapter[Posfácio]{Posfácio \subtitulo{A recepção crítica do romance}}
\hedramarkboth{Posfácio}{caio gagliardi}


\begin{flushright}
\textsc{caio gagliardi}
\end{flushright}

\noindent{}\textit{O Ateneu} é marcado por reiteradas
tentativas de enquadramento. Sem exceção, mas por razões estéticas e
formas de contextualização diversas, elas chamam a atenção para
qualidades indiscutíveis do romance. Se algumas dessas tentativas
atendem ao propósito de categorizar o texto segundo preceptivas gerais,
de fundo historiográfico, outras se fazem justamente na contramão de
uma perspectiva (e de uma didática) generalista. A dialética desse
movimento de ideias, entre o típico e o particular, marca profundamente
as escolhas e os juízos da crítica nacional. 

A começar por sua linguagem, o romance distingue"-se pelo encadeamento
rítmico das frases, que pintam uma sequência de quadros cuja descrição
pormenorizada é mais importante do que a amarração do enredo. Em geral
longos e estilizados, os períodos são ricos em imagens, comparações e
metáforas. O que Pompeia parece buscar com isso é fixar sensações no
leitor, no lugar de simplesmente contar uma história. Essas sensações
são obtidas pela transmissão de impressões, juízos e emoções do
narrador, que reconstrói através da memória seu passado individual. A
narrativa realizada pela evocação não condiz com a pretensa
objetividade do olhar realista. Em seu lugar, manifesta um caráter
acentuadamente psicológico e tom predominantemente impressionista.

No entanto, no romance, a representação da realidade escolar pode ser
tomada como uma construção alegórica que remete à realidade social,
especialmente enquanto crítica de suas instituições seletivas e
hierárquicas, legitimadas pela monarquia. Ali, Aristarco é
figura"-chave de uma crítica social realizada com tintas fortes, de
gosto naturalista. Nesse ambiente mais carregado, a sátira toma o lugar
da análise psicológica, e as personagens, reduzidas a seus instintos
primários, prestam"-se à bestialização constante.

Considerando essas facetas distintas, é identificável no
romance um regime de constante mescla e transição entre tons
contrastantes: o impressionista, de penetração psicológica, e o
expressionista, de alcance social. A quebra na unidade de tom é o traço
mais profundamente moderno de \textit{O Ateneu}, 
e que dá ensejo para muitas divergências de opinião.  

A evolução da polêmica a respeito da posição histórica
do romance, por vezes referida sem sistemática entre críticos e
pesquisadores contemporâneos, constitui um capítulo à parte na história
da recepção do romance (e possivelmente na da crítica brasileira). Isso
porque acompanhá"-la  tanto permite tornar mais aguda a compreensão do
caráter híbrido e esquivo de \textit{O Ateneu} às classificações escolares, quanto a
ondulação das tensões, isto é, das ratificações e dos revezes de que se
compõe a discussão a seu respeito. 

Nesse debate, há dois temas fundamentais em jogo: a filiação do romance
ao Naturalismo e sua leitura segundo a clave biográfica. 

O texto central da polêmica estabelece ambas as filiações, e é também a
mais dura crítica a \textit{O Ateneu}. Trata"-se do ensaio de Mário de
Andrade. Apesar de reconhecer as qualidades do texto criticado, o autor
modernista trata o romance como ``vingança'' de Raul Pompeia contra seu
internamento no colégio Abílio. O tom de acusação dessa leitura é
baseado num suposto rancor do escritor. Para Mário, \textit{O Ateneu}
revela espantosa insensibilidade de Pompeia ante a adolescência e a
amizade. O resultado é uma leitura biográfica do romance, e seu
enquadramento no Naturalismo. 

A exemplo dessa leitura, aqueles que optam por estabelecer vínculos,
seja com a vida do escritor, seja com o referido estilo de época, ainda
que reconheçam as qualidades estilísticas do romance, geralmente não o
fazem sem ressalvas. De modo diverso, aqueles que rejeitam filiações
mais estreitas ou simplesmente desviam delas, atribuem à obra juízos
menos condicionais e lhe conferem maior singularidade na história da
literatura brasileira.

Entre os leitores de ambas as tendências constam nomes consagrados da
crítica brasileira contemporânea (José Paulo Paes, José Guilherme
Merquior, Alfredo Bosi e Roberto Schwarz), da de meados do século (Otto
Maria Carpeaux, Agrippino Grieco e Lúcia Miguel"-Pereira), de
escritores"-críticos de renome (Mário de Andrade e Lêdo Ivo), além de,
com exceção ao Machado de Assis crítico literário, os três mais
importantes críticos brasileiros do século \textsc{xix} (Sílvio Romero, Araripe
Jr.~e José Veríssimo).

Para que o leitor tenha a possibilidade de acompanhar a evolução do
debate, entendeu"-se que, no lugar de uma análise mais detida,
acompanhada de resenhas críticas sobre os textos, fosse mais valioso
reproduzir aqui a voz de seus protagonistas, naqueles que se julgaram
ser os momentos principais de suas argumentações. Fique claro que os
excertos selecionados dizem respeito, ora direta, ora indiretamente, ao
debate aqui referido, e de modo algum sintetizam ou substituem a
leitura dos textos integrais, que tratam de questões variadas,
relativas ao romance, seu contexto de produção e seu autor.

A disposição desses textos, em ordem cronológica de
publicação,\footnote{ A data, disposta depois do nome de cada crítico,
refere"-se à primeira publicação do texto"-base de que se extraiu o
excerto. Muitas vezes, esses textos foram publicados em jornais,
revistas ou como prefácio para edição do romance, daí em alguns casos
ela não coincidir com a data dos volumes citados, que os reproduziram
mais tarde. A disposição de datas em sequência indica fontes distintas
do mesmo autor --- referidas no pé dos textos.} permite,
por exemplo, visualizar com clareza o peso de algumas objeções às
vinculações já referidas, como as de dois leitores notáveis, Lúcia
Miguel"-Pereira e Lêdo Ivo. Também relativizam a relevância do ensaio
de Mário, ao revelar suas fortes filiações com abordagens anteriores: a
de Veríssimo, que enquadra, com a mesma peremptoriedade de Mário,
Pompeia no Naturalismo; e a de Grieco, que, de modo muito similar à
ideia de ``vingança'', defendida pelo escritor modernista, entende
\textit{O Ateneu} como ``desforra'' do escritor. 

O leitor tem assim à disposição matéria"-prima para traçar suas
próprias associações e delas extrair outras conclusões.

\newcommand{\olho}[2]{\paragraph{\textsc{#2} [#1]}}

\olho{1888}{Tristão de Alencar Araripe Jr.}

[\ldots] o novo romancista pertence a uma classe de temperamentos
literários muito diversa da a que se filiam, na França, Balzac e Zola,
e no Brasil, A.~Azevedo. (147)

A complicação do seu temperamento literário resume"-se na seguinte
fórmula --- um realista subjetivista. Daí a sua segurança. [\ldots]

Os decadentes, os místicos, os sonâmbulos literários associam cores,
sons, aspectos vagos, isolando"-se completamente do ambiente. Raul
Pompeia apenas segrega"-se das sensações de conjunto; eis porque ele
não é um realista objetivista. (163)

\newcommand{\fontes}[1]{\begin{footnotesize}*\hspace{1ex} #1\end{footnotesize}}
%Jorge: colocar todas as ref. bl. dentro do comando \fontes{}. 
%Jorge: como no exemplo abaixo:
\fontes{\textit{Teoria, crítica e história
literária}. Seleção e apresentação de Alfredo Bosi. São
Paulo: Ed\textsc{usp}, 1978.}

\olho{1889}{Sílvio Romero} 

[\ldots] o autor d'\textit{O Ateneu} é o mais culto de seus pares no Brasil.

Não anda apenas a deglutir as migalhas da literatura
francesa. Provadamente estudioso, os clássicos latinos e gregos não lhe
metem medo, os bons autores ingleses e alemães fazem"-lhe as delícias.
Por isso ele não está preso ao naturalismo estreito e estéril da escola
de Zola, cujos romances fazem na literatura o mesmo papel dos livros de
Letourneau, Le Bom, Lefèvre \textit{et reliqui} no mundo da ciência, o papel da mediocridade
charlatanesca, esganadora e pretensiosa. Tenho medo que me atirem
pedras, ou descomposturas, mas já agora é preciso ser sincero e dizer
toda a verdade. O naturalismo de Zola, especialmente como o entendem no
Brasil, não é a última palavra em literatura. Ao lado desse
naturalismo, que se pode chamar a sistematização do mal, há um
naturalismo mais vasto, mais correto, mais exato, mais humano e mais
científico. Este conta apenas dois representantes no Brasil: Raul
Pompeia e Domício Gama.

São muito moços, começam apenas, não deram ainda toda a
medida de sua capacidade; mas, ou eu me engano muito, ou este país tem
neles dois escritores de altura acima do comum. Os outros\footnote{O
crítico se refere a Júlio Ribeiro, autor de \textit{A carne}; Horácio
de Carvalho, autor de \textit{Cromo}; Marques de Carvalho, autor de
\textit{Hortência}, e Aluísio de Azevedo, autor de \textit{O Homem}.} 
têm talento; mas esse talento não é tão maleável, tão
despreocupado, tão insinuante, tão alentado por bem dirigidos estudos.

Entretanto, Raul e Domício são hoje a minoria, representam a esquerda na
luta do naturalismo [\ldots] (254--255)

\fontes{\textit{História da literatura brasileira}.
3ª ed. aumentada. Org. e pref. por Nelson Romero. Rio de Janeiro: José Olympio, 1943.}

\olho{1907, 1912}{José Veríssimo}

Com algum exagero do ponto de vista e das proporções das coisas vistas e
escritas que acaso se lhe pode notar, o maior defeito deste livro,
senão que mais que nenhum outro de alguma forma o diminui, provando ao
mesmo tempo os anos verdes do autor, é a insignificância do assunto:
episódios da vida colegial de um menino de quinze anos. Deste efeito
resultou um outro, é que sendo a história contada em primeira pessoa,
pelo mesmo herói do romance, que fingidamente é o narrador, de fato é o
autor, homem feito, com a sua filosofia, o seu sentimento, a sua
língua, o seu estilo, quem a conta. Disto resulta uma disparidade, uma
desconformidade chocante, se me revelam a expressão. (134)

O assunto do \textit{Ateneu}
é a vida de um colegial dos onze aos quinze anos\footnote{Tendo, de
fato, ali ingressado aos onze, na verdade Sérgio passa dois anos no
colégio. As duas marcações temporais mais importantes são dadas no
3º parágrafo do capítulo \textsc{i}, ``Eu tinha onze anos'', e
no 1º parágrafo do capítulo \textsc{viii}, ``No ano
seguinte [\ldots]'' (\textsc{n}.~do \textsc{o}rg.)} num internato daquele nome
e a vida que se vivia num estabelecimento, ou generalizando --- e esta
generalização está sem dúvida no pensamento do autor --- em
estabelecimentos congêneres. Deste assunto resultam dois
inconvenientes, se não defeitos, é o primeiro que apontando a ser um
estudo ou representação de um caráter, a descrição do colegial que é o
protagonista do livro, esse estudo se faz de personagem que não tem
ainda, nem pode ainda ter um caráter, pois na sua idade o caráter ainda
não está formado; o segundo que a ficção resulta em tese preconcebida,
e todo o seu desenvolvimento obedece a esse preconceito. E
constantemente, em todos os passos do livro, sem discrepância, o autor,
homem feito, com a sua ciência da vida e o seu saber dos livros, a sua
experiência de adulto, se substitui ao narrador apenas no começo da
puberdade, qual ele o fingiu.
[\ldots] 

Porém distinto, superiormente distinto, da produção
literária do tempo. Certo é evidente nele a influência do naturalismo
francês que os romances de Eça de Queiroz vulgarizaram na nossa língua,
e que começava então a atuar na nossa literatura. Mas, ao invés do Sr.\,Aluísio de Azevedo, e de outros seguidores aqui dessa corrente
literária, Raul Pompeia apenas lhe recebeu a essência, o íntimo do
pensamento filosófico ou estético que o determinou, sem lhe adotar, se
não com grande independência, os processos e cacoetes. É nesta
autonomia de um espírito que sobrepuja as influências legítimas e ainda
consentidas do seu momento e prevalece contra elas que se há de ver o
maior testemunho da personalidade de um escritor. A personalidade de
Raul Pompeia é intensa n'\textit{O Ateneu},
que mais que um romance de escasso interesse dramático, é um compêndio
de todas as inúmeras sensações e ideias que fervilhavam àquele tempo no
cérebro em ebulição de um moço genial.  (135)

\fontes{``Raul Pompeia e \textit{O Ateneu}'' (1907). In: \textit{Últimos estudos
de literatura brasileira}. 7ª série. São Paulo:
Itatiaia/Ed\textsc{usp}, 1979.}


[\ldots] Raul Pompeia deu n'\textit{O Ateneu} a amostra mais distinta, se não a mais
perfeita, do naturalismo no Brasil. Ao contrário dos seus dois
principais êmulos nessa moda literária, Aluísio de Azevedo e Júlio
Ribeiro, que, achegando"-se demasiado ao seu figurino francês,
sacrificaram"-lhe a originalidade que acaso tinham, Raul Pompeia, com
dotes de pensador e de artista superiores aos dois, não perdeu a sua. O
seu romance é mais original e o mais distinto produto da escola aqui,
sem ser tão bem composto como os melhores de Aluísio de Azevedo. 

\fontes{\textit{História da literatura brasileira} -- de Bento Teixeira 
(1601) a Machado de Assis (1908). 7ª ed., São Paulo: Topbooks, 1998.}

\olho{1933}{Agrippino Grieco} 

N'\textit{O Ateneu}, como no
\textit{Louis Lambert}, de Balzac, há as
confidências de um rapaz enjaulado num internato. É o eterno livro
autobiográfico em que os autores, falando na primeira pessoa, sem
nenhuma hipocrisia, dão o melhor de si mesmos. Confissões algo cínicas,
tais narrações assumem quase sempre um caráter de vindicta, de
represália. No caso, é a desforra, pelo sarcasmo, das humilhações de
uma pedagogia enfática e burlesca. [\ldots] Grandes dias do colégio,
festas cívicas, discursos, patriotadas, tudo foi admiravelmente colhido
por esse amargo humorista que teve tantas palavras esbofeteantes para a
vaidade dos falsos pedagogos. Pompeia é o primeiro talvez dos
impressionistas da nossa prosa. Influenciado possivelmente pelos
Goncourt, sente"-se"-lhe o sistema nervoso à mostra. (101--102).

\fontes{\textit{Evolução da prosa brasileira}. Rio de Janeiro: Ariel, 1933.}

\olho{1941}{Mário de Andrade}

Já se disse que \textit{O Ateneu} é o menos naturalista dos nossos romances do
Naturalismo. Não penso assim. Ele representa exatamente os princípios
estético"-sociológicos, os elementos e processos técnicos do
Naturalismo. É sempre aquela concepção pessimista do homem"-besta,
dominado pelo mal, incapaz de vencer os seus instintos baixos --- reflexo
dentro da arte das doutrinas evolucionistas. É sempre aquele exagerar
inconsciente e ao sério das manifestações destrutivas do ser, baseado
numa psicologia do terror, que concebe os homens como bestas e ignora a
``parte do anjo''. É sempre aquela crítica ardorosa e deformadora das
formas sociais mal ajustadas e infamantes que, contrastando
romanticamente com o pessimismo evolucionista, acredita na melhoria do
ser e num futuro mundo ideal --- novo avatar de romantismo que apenas
substitui a imagem lírica e sentimental pelas imagens igualmente
sentimentais do abjeto. Se ainda existem visões de delicadeza no livro,
elas derivam muito mais do próprio assunto que de uma fuga anti ou
extra"-naturalista do autor. E este transvazou o seu temperamento na
obra, e de maneira dolorosíssima, se demonstrando incapaz do exercício
da amizade e, consequentemente, de uma cruel incompreensão ante a
adolescência. E quanto à expressão, ecoa no Brasil, e de maneira
singularmente brilhante e eficaz, a ``écriture artiste'' aparecida no
naturalismo francês. E português também, pois carece não esquecer que
Eça já aparecera, e nos tempos da publicação
d'\textit{O Ateneu}, dava nos folhetins cariocas
da \textit{Gazeta de Notícias} \textit{A Relíquia}. \textit{O
Ateneu} não é menos naturalista que os seus êmulos
brasileiros. E admiravelmente, com hábil consciência técnica, Raul
Pompeia soube ajustar a brutalidade de escola ao seu assunto, que era,
por natureza, menos brutal. Porém, mesmo assim, não deixou de botar
inutilmente no livro um assassínio e um incêndio. \textit{O
Ateneu} representa um dos aspectos particulares mais
altos do Naturalismo brasileiro. (184)  

\fontes{``\textit{O Ateneu}'' (1941). In: \_\_\_. \textit{Aspectos da literatura
brasileira}. 5ª ed., São Paulo: Livraria Martins, 1974.}

\olho{1950}{Lúcia Miguel"-Pereira}

Por isso tudo, pelo seu sentido psicológico, pelo seu
tal ou qual estetismo, é que o romance não me parece enquadrar"-se no
Naturalismo. A este pediu Pompeia emprestados alguns recursos, mas para
empregá"-los a seu jeito. A objetividade, por exemplo, que serviu para
melhor desentranhar a vida interior de Sérgio; a ação do meio que,
entretanto, em vez de modificar o menino, fortalece"-o no seu
alheamento. Um escritor naturalista nunca teria desdenhado dos lances
escabrosos que poderia fornecer o tema d'\textit{O
Ateneu}, nunca se teria contentado com insinuações e
meia palavras. [\ldots] Raul Pompeia visava apenas aos conflitos e
problemas interiores; para situá"-los é que recorreu às descrições
detalhadas; não quis explicar Sérgio pelas reações do seu temperamento
em face do internato, mas criou, sem se valer dos dogmas em moda, um
menino inadaptado, prisioneiro do próprio eu, cujo caminho ninguém encontrou.

Sem dúvida, há n'\textit{O Ateneu} a realidade, narrada com aquela minúcia que
Flaubert exigia do seu discípulo Maupassant; mas tendo os nossos
realistas se encerrado no âmbito do naturalismo de Zola, e assim
constituído uma escola literária muito coesa e característica tanto na
técnica como no espírito, não se pode incluir entre eles Raul Pompeia.
Se fosse um naturalista e não, mais largamente, um observador da
realidade, ele teria encarado os problemas psicológicos de Sérgio como
corolários dos dados fisiológicos. Ao contrário, não faz o feitio de
Sérgio depender das suas condições físicas e, se apresenta a
homossexualidade como quase geral entre rapazes privados de contatos
femininos, não a explica em termos biológicos; aliás, não explica nunca
coisa alguma; narra apenas, sem fornecer motivos nem tirar conclusões.
[\ldots]

Só entendendo"-se \textit{O Ateneu} como um romance de tese, destinado a provar a
má influência dos internatos, é que se poderia encaixá"-lo no
naturalismo. Mas isso equivaleria a inverter o livro e diminuí"-lo,
colocando o acessório no lugar do principal. Vendo Sérgio em si mesmo,
e não em função do meio, embora este seja estudado, Raul Pompeia
rejeitou as leis do romance experimental, às quais obedecem
passivamente os verdadeiros naturalistas. [\ldots]

Raul Pompeia, não estabelecendo premissas nem chegando a
conclusões, não fazendo do seu herói um fruto exclusivo do ambiente,
deixando, ao contrário, perceber que a mudança de vida pouco lhe
alterará o feitio pessoal, guardou a sua liberdade de criação. [\ldots]
Acresce ainda, para colocar \textit{O Ateneu} fora do naturalismo, o seu aspecto
autobiográfico, mais importante para a compreensão de Raul Pompeia
do que do livro que vive por si, independente do autor, mas nem por
isso desprezível, sobretudo para captar"-lhe o sentido. Este foi o das
confissões, caras ao românticos, e odiosas ao seus opositores. Não se
concebe um naturalista ortodoxo partindo do próprio eu, de impressões
subjetivas, para a construção de uma obra, embora a revestisse de
minúcias subjetivas. (112--114)

\fontes{\textit{Prosa de ficção 1870--1920}. Rio de Janeiro: José Olympio, 1950.}

\olho{1951}{Otto Maria Carpeaux}

Alguns contemporâneos do naturalismo e do parnasianismo
resistem a qualquer tentativa de classificação. Na época,
\textit{O Ateneu} foi considerado romance
naturalista, e muitos repetem, até hoje, esse lugar"-comum. Mas
\textit{O Ateneu} é romance de interpretação
psicológica, sem revelar, no entanto, semelhança alguma com os
romances psicológicos de Machado de Assis. Na verdade, Raul Pompeia é
figura isolada [\ldots] \textit{O Ateneu} é
romance impressionista [\ldots] (171)

\fontes{\textit{Pequena bibliografia crítica da literatura brasileira}. Rio de
Janeiro: \versal{MEC}, 1951.}

\olho{1956}{Brito Broca}

O ano de 1888, em que surgiu \textit{O
Ateneu}, foi uma grande data para o Naturalismo no
Brasil: nele apareceram mais cinco romances moldados pela escola de
Zola --- \textit{O missionário}, de Inglês
de Souza; \textit{Cromo}, de Horácio de
Carvalho; \textit{Hortênsia}, de Marques
Carvalho; \textit{O lar}, de Pardal Mallet;
e \textit{A carne}, de Júlio Ribeiro. Com
exceção do primeiro, todos obras inferiores, mas publicadas com certo
rumor, por trazerem a marca do escândalo. Dessa marca não estava também
isento \textit{O Ateneu}, que com eles
apresentava alguns pontos de contato. De onde a classificação de
romance naturalista que lhe foi erroneamente aplicada e aceita por
muitos críticos, inclusive José Veríssimo.

Logo após o lançamento do livro, Araripe Júnior, numa
série de artigos no jornal \textit{Novidades} (dezembro de 1888 a
janeiro de 1889), distinguira o seu caráter psicológico. Não obstante,
é preciso considerar: embora romance psicológico, \textit{O Ateneu} 
denuncia influências naturalistas. Sem o exemplo de Zola, Raul Pompeia 
não se animaria a passar por ``certas escabrosidades'' --- 
para usarmos dos termos da nota de apresentação na 
\textit{Gazeta de Notícias} --- mesmo ``com a delicadeza 
e o fino tato de um artista da raça''.

[\ldots]

Mário de Andrade, que insistiu em considerá"-lo um romance naturalista,
apesar da agudeza crítica com que o apreciou sob outros aspectos, viu
ainda uma concessão à ``brutalidade da escola'' no assassinato e no
incêndio que Raul Pompeia ``botou inutilmente no livro''. Ora, o
incêndio, como já observamos, devia ter obedecido a razões
psicológicas. E quanto ao assassinato, não nos parece uma concessão ao
Naturalismo e sim ao leitor habituado ao gênero folhetinesco. Pompeia
sentira, decerto, a necessidade de introduzir um acontecimento dessa
natureza para ferir um pouco a atenção do público. Pois, convenhamos, o
romance, mesmo aos ``homens das letras, aos amadores das artes'' aos
quais foi entregue ``confiadamente'', nem sempre é de leitura amena. Num
incessante trabalho de estilo, Pompeia abusa das metáforas, pensa por
meio de imagens e se estende às vezes em divagações um tanto
fatigantes, por demasiado retorcidas. Mas é o estilo, apesar dessas
falhas, que dá ao livro um cunho essencialmente artístico e literário,
tornando"-o uma das grandes obras de nossa ficção. (224--225)

\fontes{``Raul Pompeia''. In \textit{Ensaios de mão canhestra}. São Paulo: Polis,
1981.}

\olho{1958}{Eugênio Gomes}

A situação de Raul Pompeia, que adquiriu tamanho renome com um só livro,
torna imediatamente lembrado o fenômeno Manuel Antônio de Almeida, com
quem se identifica ainda por sua natural resistência às classificações
literárias. Parnasianismo? Realismo? Naturalismo? Impressionismo? Para
qual pendeu? Embora o Impressionismo seja o que melhor caracteriza sua
posição definitiva, não foi nada insensível às demais correntes
estéticas contemporâneas. (131)

[\ldots] o detalhe, a mímica, os gestos, os tiques, o particular, passava a
ter mais importância do que o geral. Era enfim a subversão de valores
da narrativa, em cuja fixação o pincel fino ou o bico da pena do
miniaturista se substituía à brocha gorda do naturalismo à Zola. (143)

\fontes{\textit{Aspectos do romance
brasileiro}. Bahia: Publicações da Universidade da Bahia, 1958.}

\olho{1959}{Afrânio Coutinho}

Em posição singular, estão Machado de Assis e Raul Pompeia, os quais,
não obstante revelarem aqui e ali impregnações naturalistas, são
realistas independentes, no  caso de Pompeia posta em realce essa
independência pelos entretons impressionistas que marcam peculiarmente
a sua obra.  (198)

No Brasil, a primeira grande repercussão do Impressionismo é em Raul
Pompeia. Discípulo dos Gongourt, adepto da ``écriture artiste'' e da
prosa poética, depois de formar o espírito na doutrina do Naturalismo,
recebeu a influência da estética simbolista e só encontrou plena e
satisfatória expressão dentro dos cânones do Impressionismo. (228)

\fontes{\textit{Introdução à literatura no Brasil}.
7ª ed. Rio de Janeiro, Editora Distribuidora de Livros Escolares Ltda, 1972.
(1959)}

\olho{1960}{Temístocles Linhares}

Ainda que enquadrado no romance psicológico,
\textit{O Ateneu} não fugia ao naturalismo.
[\ldots] É preciso considerar ainda que o naturalismo, defendido por Zola,
nem por ele próprio chegou a ser seguido com plena e lógica coerência.
Há até quem sustente que ele pode ser grande escritor por ter muitas
vezes contrariado as suas ideias preferidas.

Ora, se tal se deu com o próprio Zola, sem dúvida o maior teórico e
representante da escola, não seria justo exigir de Pompeia o
cumprimento à risca de seus postulados.

A sua posição, em certo sentido, se aproximava até mais
de outros naturalistas. Falou"-se muito nos Goncourt, adeptos da
``prosa artística''. Mas nem sempre a aproximação tinha cabimento. Começa
que em \textit{O Ateneu} não havia uniformidade estilística. A preocupação esteticista não foi exclusiva
em sua composição. Talvez outra aproximação tivesse mais cabimento: a
com Huysmans, na sua fase naturalista e cujo naturalismo consistia
menos na observação da realidade do que no ódio da realidade. Por
conseguinte, numa visão unilateral, determinada por um pessimismo quase
radical. (16--18)

\fontes{``Apresentação''. In \textit{Raul Pompeia} -- \textit{Trechos escolhidos}.
Rio de Janeiro: Agir, 1960.}

\olho{1963}{Lêdo Ivo}

Os manuais de história da literatura são contagiosos --- a maneira de ver
ou de desver, de Veríssimo, atravessou como um rastilho os nossos
ciclos críticos, e ainda encontra prosélitos.

Um desses foi Mário de Andrade que, tendo se detido
diante de \textit{O Ateneu} nada menos de
cinquenta e seis anos após a sua aparição, não soube beneficiar"-se da
nova perspectiva histórico"-cultural posta a seu favor. Em seu
trabalho --- que desenvolve uma celebrosa teoria de que o móvel por assim
dizer exclusivo da altíssima criação de Pompeia foi o propósito
indiscriminado de vingança --- Mário de Andrade não se mostra à altura de
sua habitual sagacidade crítica [\ldots].

[\ldots]

Com a sua tese inaceitável que escamoteia, dengosamente,
a verdade e a arte de \textit{O Ateneu}, o
ensaio de Mário de Andrade mereceria uma longa e desdobrada refutação
[\ldots]. (20--21)

[\ldots]

Não podemos, porém deixar de proclamar que, no tocante à
filiação estética do autor de \textit{O Ateneu}, a razão pende para os que lhe averbaram o
inconformismo à corrente naturalista --- muito embora seja imperioso
fazer"-se a distinção necessária entre o seu processo impressionista e
a sua visão simbólica, que não devem ser confundidos. Contudo, os que
sublinharam o caráter impressionista de sua prosa estabeleceram uma
distinção imprescindível. Além do mais, a inclusão de Pompeia no rol
dos naturalistas de cama e mesa caracteriza escandaloso desconhecimento
dos postulados da escola, não só em sua erupção na França, como em sua
transplantação para o Brasil. 

Os elementos ideológicos com que era feito o romance
naturalista, segundo a receita de Zola --- a fé na ciência e no
progresso, o determinismo, a animalidade, a hereditariedade, a prática
experimental --- estão praticamente ausentes n'\textit{O Ateneu}. (26)

[\ldots]

Liberto do pitoresco, da efusão nativista, dos
regionalismos indiscretos, do paisagismo primaveril e do frágil
psicologismo dos românticos, é o segundo romance brasileiro em que a
inteligência vigilante predomina sobre as gratuidades
consentidas,\footnote{O primeiro é \textit{Memórias póstumas de Brás
Cubas}, de Machado de Assis.} a se voltar para o
mistério da alma humana, a promover o cotejo entre a solidão individual
e a sociedade, e a trazer, em sua linguagem e em seu enredo, uma
dramática visão do universo. (25--26)  

\fontes{\textit{O universo poético de Raul Pompeia}. Rio de Janeiro: Livraria
São José, 1963.}

\olho{1965}{Roberto Schwarz}

Mário de Andrade faz, em bonito estudo, uma discussão
larga e minuciosa do cunho vingativo de \textit{O
Ateneu}. A orientação biográfica de seu ensaio oculta,
contudo, a vingança presente \textit{no
romance}, pois o sentimento é visto noutro plano,
enquanto relação psicológica de Raul Pompeia com seu livro. O
biografismo crítico, preso à ideia do todo contínuo formado por autor e
obra, tende a interpretar
\textit{distribuindo}: o
\textit{subjetivismo}, dado no tom e nas
imagens, ilumina a psicologia do
\textit{criador}; os
\textit{fatos}, por sua vez, usam"-se para
estabelecer o \textit{conteúdo} da criação.
Consequência é o empobrecimento do texto, pois o que nele se
objetivara, passando a ser parte sua, é visto como atributo do autor,
ser vivo e inesgotável no papel impresso. Mesmo um excelente ensaio
como o de Mário de Andrade não escapa a esse quadro, que rouba ao
romance de Raul Pompeia, a nosso ver, uma das dimensões mais modernas,
a superação do Realismo pela presença emotiva de um narrador. (25) 

\fontes{``\textit{O Ateneu}''. In:
\_\_\_. \textit{A sereia e o desconfiado}.
Rio de Janeiro: Civilização Brasileira, 1965.}

\olho{1964}{Nelson Werneck Sodré}

O aparecimento, em folhetim da \textit{Gazeta
de Notícias}, em 1888, do romance \textit{O
Ateneu}, não poderia deixar de constituir uma
surpresa. O naturalismo estava em pleno desenvolvimento e, embora
tivesse uma vigência curta, exerceu influência poderosa, a que não
fugiu o próprio Pompeia. Não fugiu nem mesmo nesse estranho e por vezes
fascinante livro de reminiscências, com o seu tom amargo e ao mesmo
tempo saudosista de quem se refugia na infância. Realizado com um
capricho minucioso, e ainda assim ardente no que exprime e no que
sugere, o romance de Pompeia permanece um recanto isolado da ficção
brasileira, mesmo consideradas as obras muito posteriores, trabalhadas
em outro sentido, e em que os memorialistas se substituem aos
ficcionistas. A prosa contida, rigorosa, um tanto caprichada do autor
não esconde a intensa vibração de algumas cenas e a grave crise de
sensibilidade que constitui o fundo do problema apresentado.
\textit{O Ateneu} permanece isolado, em
nossas letras. Revelara, entretanto, um escritor de primeira ordem, que
os trabalhos anteriores vinham escondendo e que se confirmaria em
algumas páginas perdidas na imprensa, cheias de colorido, de vivacidade
e de paixão. (502)

\fontes{\textit{História da literatura brasileira} -- seus fundamentos econômicos.
4ª ed. Rio de Janeiro: Ed. Civilização Brasileira, 1964.}

\olho{1970, 2003}{Alfredo Bosi}

Raul Pompeia partilhava com Machado o dom do
memorialista e a finura da observação moral, mas no uso desses dotes
deixava uma tal carga de passionalidade que o estilo de seu único
romance realizado, \textit{O Ateneu}, mal
pode definir, em sentido estrito, realista; e já houve quem o dissesse
impressionista, afetado pela plasticidade nervosa de alguns retratos e
ambientações, por outras razões se poderiam nele ver traços
expressionistas, como o gosto do mórbido e do grotesco com que deforma
sem piedade o mundo do adolescente. (203--4)

[\ldots]

Não fora o seu talento excepcional de artista, Raul Pompeia teria
naufragado no puro romance de tese. Aos naturalistas típicos, que lhe
eram inferiores como estilistas, não foi poupada a armadilha. (208) 

\fontes{\textit{História concisa da literatura brasileira}.  3ª ed. São Paulo: Cultrix,
1982 (1970)}


A poética de Raul Pompeia é um dos muitos exemplos dessa
passagem que envolveu toda a arte ocidental no último quartel dos
Oitocentos. Do Naturalismo ao Impressionismo. Sempre a
\textit{conversão do Naturalismo}, segundo a
fórmula incisiva que Otto Maria Carpeaux cunhou para qualificar a
viragem do romance em Dostoiévski e em Tolstói, e a metamorfose do
drama em Strindberg e em Ibsen. A expressão colhe alguns signos de
mudança, no espírito e na forma, que se observam na literatura europeia
e, em parte, na brasileira de entre"-séculos.

A ``conversão'', termo que implica dialetizar um estilo de pensar e dizer
já esgotado, aparece hoje como emblema da modernidade. (79)

\fontes{ ``\textit{O Ateneu}, opacidade e destruição''. In: \_\_\_. \textit{Céu, inferno}
-- ensaios de crítica literária e ideológica. São Paulo: Duas Cidades/Ed. 34,
2003.}

\olho{1977}{José Guilherme Merquior}

A prosa não naturalista do fim do século se colocava
sob o signo da lembrança. Os romances"-``memórias'' de Machado, o
memorialismo de Nabuco e o tradicionalismo de Eduardo Prado manifestam
em comum uma forte sensibilidade ao tempo e ao passado. O livro do
nosso maior romancista impressionista depois de Machado de Assis --- 
\textit{O Ateneu} [\ldots] --- traz como subtítulo ``crônica de saudades''. (191)

[\ldots]

Pompeia é uma das melhores penas satíricas de nossa literatura.

[\ldots]

Pois \textit{O Ateneu} tem muito de discussão ideológica; 
chega a ser um pequeno romance"-ensaio.
(193) 

\fontes{ \textit{De Anchieta a Euclides} -- breve história da literatura brasileira.  Rio
de Janeiro: José Olympio, 1977.}

\olho{1983}{José Paulo Paes}

Coube a Raul Pompeia introduzir entre nós a escrita
artística ou a ``prosa de arte'', como lhe chamam os italianos. Nas
anotações íntimas que deixou, há referências explícitas aos Goncourt
assim como críticas à ``expressão fria'' de Mérimée e à falta de ritmo
da prosa ``sem forma literária'' de Stendhal. Em vez da neutralidade
stendhaliana, copiada da do Código Civil, preconizava Pompeia ``o
processo original de dizer --- a eloquência própria'' de cada escritor,
visto que ``a prosa tem de ser eloquente para ser artística, tal qual os
versos'' e que ``o grande fator do pitoresco, da prosa como do verso, são
as imagens no ritmo''. Tal concepção teórica de uma quase indistinção 
entre poesia e prosa, ele a levou à prática não só nas
\textit{Canções sem metro}, em que o
martelamento silábico dos versos é substituído pela flexibilidade
rítmica do poema em prosa, como n'\textit{O Ateneu}, 
onde a frequência da metáfora e a riqueza
inventiva do adjetivo configuram uma prosa de cunho ornamental, bem
diversa, nisso, da discrição da prosa machadiana sua coeva. Mas
ornamento, no caso, não é acréscimo nem excrescência gratuita; é
estilização consubstancial, organicamente ligada ao empenho de
caricatura d'\textit{O Ateneu}, pelo que,
conquanto este tenha sido publicado antes da voga artenovista entre
nós, se pode vê"-lo como seu precursor no campo da prosa de ficção.
(72)

\fontes{ ``O art nouveau na literatura brasileira''. In: \textit{Gregos e baianos} --
ensaios. São Paulo: Brasiliense, 1985.}


\begin{bibliohedra}
\tit{Abreu}, Capistrano de. \textit{Ensaios e estudos}.
1ª série. Rio de Janeiro: Edição da Sociedade
Capistrano de Abreu, 1931.

\tit{ANDRADE}, Mário de. ``\textit{O Ateneu}''. In: \_\_\_. \textit{Aspectos da
literatura brasileira}. 5ª. ed. São Paulo: Livraria
Martins, 1974.

\tit{Araripe Jr.}, Tristão de Alencar. ``Raul Pompeia,
\textit{O Ateneu} e o romance psicológico'' e ``Raul Pompeia como
esteta''. In: \_\_\_. \textit{Obra crítica}. Vols. \textsc{ii} e \textsc{iii}. Rio de
Janeiro: Casa de Rui Barbosa, 1960. Também em: \textit{Teoria, crítica
e história literária}. Sel. e apresentação de Alfredo Bosi. São Paulo:
Ed\textsc{usp}, 1978.  

\tit{Bosi}, Alfredo. ``\textit{O Ateneu}, opacidade e
destruição''. In: \_\_\_. \textit{Céu, inferno} -- ensaios de crítica
literária e ideológica. São Paulo: Duas Cidades / Ed.~34, 2003.

\titidem. ``Raul Pompeia''. In: \_\_\_. \textit{História concisa da
literatura brasileira}. 3ª. ed. São Paulo: Cultrix,
1982.

\tit{Broca}, Brito. ``Raul Pompeia''. In: \_\_\_. 
\textit{Ensaios de mão canhestra}. São Paulo: Polis, 1981.

\tit{Gomes}, Eugênio. ``Raul Pompeia''. In: \_\_\_.
\textit{Aspectos do romance brasileiro}. Bahia: Publicações da
Universidade da Bahia, 1958.

\titidem. ``Pompeia e a natureza''. In: \_\_\_. \textit{Visões e
revisões}. Rio de Janeiro: INL, 1958.

\tit{Grieco}, Agrippino. \textit{Evolução da prosa brasileira}. Rio
de Janeiro: Ariel, 1933

\tit{Heredia}, José López. \textit{Matéria e forma narrativa
d'}O Ateneu. São Paulo: Quíron, 1979.

\tit{IVO}, Lêdo. \textit{O universo poético de Raul Pompeia}.
Rio de Janeiro: Livraria São José, 1963.

\titidem. ``Raul Pompeia: o desastre universal''. In: \_\_\_.
\textit{Teoria e celebração}. São Paulo: Duas Cidades, 1976.

\tit{LINHARES,} Temístocles. ``Apresentação''. In \versal{POMPEIA}, Raul.
\textit{Trechos escolhidos}. Rio de Janeiro: Agir, 1960.

\tit{MERQUIOR}, José Guilherme. \textit{De Anchieta a Euclides} -- breve
história da literatura brasileira. Rio de Janeiro: José Olympio, 1977.

\tit{MIGUEL"-PEREIRA}, Lúcia. \textit{Prosa de ficção 1870--1920}. Rio de
Janeiro: José Olympio, 1950. 

\tit{PAES}, José Paulo. ``Sobre as ilustrações d'\textit{O
Ateneu}''. In: \_\_\_. \textit{Gregos \& Baianos} -- ensaios. São Paulo:
Brasiliense, 1985.

\tit{PERRONE"-MOISÈS}, Leyla (org.). \textit{O Ateneu: retórica e paixão.}
São Paulo: Brasiliense/ Ed\textsc{usp}, 1988.

\tit{Pontes}, Eloy. \textit{A vida inquieta de Raul Pompeia}.
Rio de Janeiro: José Olympio, 1935.

\tit{Reis}, Zenir Campos. ``Opostos, mas justapostos''. In
\versal{POMPEIA}, Raul. \textit{O Ateneu}. 13ª. ed. São Paulo:
Ática, 1991.

\tit{SANTIAGO}, Silviano. ``\textit{O Ateneu}: contradições e perquirições''.
In: \_\_\_. \textit{Uma literatura nos trópicos}. São Paulo:
Perspectiva, 1978.

\tit{SCHWARZ}, Roberto. ``\textit{O Ateneu}''. In: \_\_\_. \textit{A sereia e o
desconfiado}. Rio de Janeiro: Civilização Brasileira, 1965. 

\tit{Silveira}, Francisco Maciel. ``Introdução''. In:
\textsc{Pompeia}, Raul. \textit{O Ateneu}. São Paulo: Cultrix, 1976.

\tit{Stegagno"-Picchio}, Luciana. ``Raul Pompeia: romanzo
psicologico e prosa impressionista''. In: \_\_\_. \textit{La letteratura
brasiliana}. Firenze: Sansoni Accademia, 1981.

\tit{TORRES}, Artur de Almeida. \textit{Raul Pompeia} -- estudo
psicoestilístico. Rio de Janeiro: Livraria São José, 1972.

\tit{VERÍSSIMO}, José. \textit{História da literatura
brasileira } -- de Bento Teixeira (1601) a Machado de
Assis (1908). 7ª 
ed. São Paulo: Topbooks, 1998. 

\titidem. ``Raul Pompeia e \textit{O Ateneu}''. In: \_\_\_.
\textit{Últimos estudos de literatura brasileira}. 7ª
série. São Paulo: Itatiaia/Ed\textsc{usp}, 1979.
\end{bibliohedra}